% !TEX root = tesis.tex

Obtener y procesar datos demográficos y sociológicos fueron uno de los procesos más importantes para entender fenómenos que afectan a toda la población desde por lo menos el Siglo XVII~\cite{friendly2006}, y encontrar formas simples e intuitivas de visualizarlos tiene un gran impacto en nuestra manera de entender los datos~\cite{minard1844,snow1855}. Formas comunes de obtener datos cuantitativos de estratificación económica usualmente involucran investigación de archivos o encuestas sociales~\cite{bulmer1977},
% ; sin embargo esos métodos no pueden presentar datos que sean simultáneamente exactos, actualizados, y que aplican a una población grande sin depender de métodos estadísticos.
y dependen de métodos estadísticos.

Las operadoras de telecomunicaciones (``telcos'') tienen acceso a una gran cantidad de información sobre las comunicaciones y hábitos de sus usuarios~\cite{huurdeman2003}, pero la habilidad de guardar y procesar esos datos ha dado grandes pasos en los últimos años gracias a nuevas y más poderosas computadoras y técnicas de minería de datos. Lo mismo puede decirse sobre la información sociológica y económica contenida por bancos y tarjetas de crédito, y por la relación entre estas dos fuentes de datos.

La minería de datos de telcos a gran escala es un área relativamente nueva que se usa principalmente para aplicaciones internas~\cite{han2002emerging}, pero la gran cantidad de información sociológica es de gran interés para temas académicos relacionados a la sociología. Esta tesis se basa en métodos usaros por Óskarsdottir et.\ al.~\cite{oskarsdottir2016} y Singh et.\ al~\cite{singh2013predicting}, además de una fuente de información de una telco y de un banco grande para encontrar que la distribución de ingresos de los usuarios sigue de manera cercana (pero no exacta) la distribución de ingresos de la población en general.

Hay una fuerte homofilia entre los ingresos de contactos en la telco, que se usa junto con la distribución desigual de dinero en la población para crear una metodología, basada en estadística bayesiana, para inferir el nivel socioeconómico de un gran subconjunto de usuarios en la red sin información bancaria con $\AUC = 0.746$. El método bayesiano es luego comparado con otros métodos basados en aprendizaje automático supervisado para probar que, aunque toma menos información de entrada, es un mejor predictor de características sociales en este tipo particular de red.
