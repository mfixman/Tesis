\documentclass[runningheads]{dalthesis/dalthesis}

\usepackage{amsfonts}
\usepackage{amsmath}
\usepackage{amssymb}
\usepackage[]{changepage}
\usepackage[compatibility=false]{caption}
\usepackage{float}
\usepackage[T1]{fontenc}
\usepackage[]{graphicx}
\usepackage[utf8]{inputenc}
\usepackage{makecell}
\usepackage{multirow}
\usepackage{pbox}
\usepackage{relsize}
\usepackage[group-separator={,},alsoload=binary]{siunitx}
\usepackage{soul}
\usepackage{tabularx}
\usepackage{booktabs}
\usepackage{tabu}
\usepackage[normalem]{ulem}
\usepackage[table,dvipsnames]{xcolor}
\usepackage{epigraph}
\usepackage{multirow}
\usepackage{makecell}
\usepackage{xfrac}
\usepackage{exscale}
\usepackage{subcaption}
\usepackage{capt-of}
\usepackage[perpage]{footmisc}
\usepackage[hidelinks=true]{hyperref}
\numberwithin{equation}{section}
\usepackage[capitalise, nameinlink, noabbrev]{cleveref}
\usepackage{enumitem}
\usepackage{dot2texi}
\usepackage[export]{adjustbox}
\usepackage{mathtools}
\usepackage{microtype}
\usepackage{fancyhdr}
\usepackage{textcase}

\usepackage{tikz}
\usetikzlibrary{shapes}
\usetikzlibrary{arrows}
\usetikzlibrary{bayesnetWLee}

\bibliographystyle{alpha}

\newcommand{\todo}[1]{\textbf{\color{red} TODO:\@ #1}}
\newcommand{\maybe}[1]{\footnote{\color{red} #1}}

\newcommand{\noimage}[1]{%
  \setlength{\fboxsep}{-\fboxrule}
  \framebox(300, 100){\texttt{#1}}
}
\newcommand{\includegraphicsmaybe}[1]{\IfFileExists{#1}{\includegraphics[width=\textwidth, height=.4\textheight, keepaspectratio]{#1}}{\noimage{#1}}}

\newcommand{\closeopen}[2]{\left[ #1, #2 \right)}
\newcommand{\mathmuchlarger}[1]{\mathlarger{\mathlarger{#1}}}

\newcommand{\Beta}{B}
\newcommand{\mathA}{\mathcal{A}}
\newcommand{\mathB}{\mathcal{B}}
\newcommand{\mathC}{\mathcal{C}}
\newcommand{\mathG}{\mathcal{G}}
\newcommand{\mathP}{\mathcal{P}}

\newcommand{\dsfrac}[2]{\frac{\displaystyle{} #1}{\displaystyle{} #2}}
\newcommand{\NA}{---}

\newcommand{\ct}[1]{\multicolumn{1}{c}{#1}}

\setcounter{secnumdepth}{3}
\captionsetup[table]{skip = 2ex}

\setlength{\tabcolsep}{3pt}

\renewcommand*{\thefootnote}{\fnsymbol{footnote}}

\DeclareMathOperator{\ypred}{y_{pred}}
\DeclareMathOperator{\ytrue}{y_{true}}

\DeclareMathOperator{\rank}{rank}
\DeclareMathOperator{\cov}{cov}

\DeclareMathOperator{\TP}{TP}
\DeclareMathOperator{\TN}{TN}
\DeclareMathOperator{\FP}{FP}
\DeclareMathOperator{\FN}{FN}
\DeclareMathOperator{\FPR}{FPR}
\DeclareMathOperator{\AUC}{AUC}
\DeclareMathOperator{\TNR}{TNR}
\DeclareMathOperator{\TNA}{TNA}
\DeclareMathOperator{\TPA}{TPA}
\DeclareMathOperator{\TPR}{TPR}

\DeclareMathOperator{\Precision}{Precision}
\DeclareMathOperator{\Recall}{Recall}
\DeclareMathOperator{\InvPrecision}{Inverse\ Precision}
\DeclareMathOperator{\InvRecall}{Inverse\ Recall}
\DeclareMathOperator{\Accuracy}{Accuracy}

\DeclareMathOperator{\calls}{calls}
\DeclareMathOperator{\etime}{time}
\DeclareMathOperator{\sms}{sms}
\DeclareMathOperator{\contacts}{contacts}

\DeclareMathOperator{\ein}{in}
\DeclareMathOperator{\out}{out}

\DeclareMathOperator{\low}{low}
\DeclareMathOperator{\high}{high}

\DeclareMathOperator{\train}{train}
\DeclareMathOperator{\test}{test}

\DeclareMathOperator{\Betainc}{\Beta_{\operatorname{inc}}}

\DeclareMathOperator{\incalls}{incalls}
\DeclareMathOperator{\outcalls}{outcalls}
\DeclareMathOperator{\insms}{insms}
\DeclareMathOperator{\outsms}{insms}
\DeclareMathOperator{\intime}{outtime}
\DeclareMathOperator{\outtime}{outtime}
\DeclareMathOperator{\incontacts}{incontacts}
\DeclareMathOperator{\outcontacts}{outcontacts}

\DeclareMathOperator{\incallslow}{incallslow}
\DeclareMathOperator{\outcallslow}{outcallslow}
\DeclareMathOperator{\insmslow}{insmslow}
\DeclareMathOperator{\outsmslow}{insmslow}
\DeclareMathOperator{\intimelow}{outtimelow}
\DeclareMathOperator{\outtimelow}{outtimelow}
\DeclareMathOperator{\incontactslow}{incontactslow}
\DeclareMathOperator{\outcontactslow}{outcontactslow}

\DeclareMathOperator{\incallshigh}{incallshigh}
\DeclareMathOperator{\outcallshigh}{outcallshigh}
\DeclareMathOperator{\insmshigh}{insmshigh}
\DeclareMathOperator{\outsmshigh}{insmshigh}
\DeclareMathOperator{\intimehigh}{outtimehigh}
\DeclareMathOperator{\outtimehigh}{outtimehigh}
\DeclareMathOperator{\incontactshigh}{incontactshigh}
\DeclareMathOperator{\outcontactshigh}{outcontactshigh}

\DeclareMathOperator{\neigh}{Neigh}

\DeclareMathOperator{\dir}{dir}
\DeclareMathOperator{\cat}{cat}

\DeclareMathOperator{\ego}{ego}

\DeclareMathOperator{\logit}{logit}


\title{Inference of Socioeconomic Status\\in a Communications Graph:\\A Bayesian Approach}
\author{Martín Fixman}
\date{Buenos Aires, 2017}

\degree{Licenciatura en Ciencias de la Computación}
\degreeinitial{L.C.C.}
\faculty{Ciencias Exactas y Naturales}
\dept{Facultad de Ciencias Exactas y Naturales}
\defencemonth{June}\defenceyear{2013}

\university{Universidad de Buenos Aires}
\address{Buenos Aires, Argentina}

\nolistoftables
\nolistoffigures
\notitlepage

\def\tituloTesis{Inference of Socioeconomic Status \\ in a Communications Graph: \\ A Bayesian Approach}
\def\runtitulo{Inference of Socioeconomic Status in a Communications Graph}
\def\director{Esteban Feuerstein}
\def\codirector{Carlos Sarraute}

\dedicate{To the time I spent thinking of a dedication to use as a placeholder.}

\setlength{\headheight}{15pt}
\lhead[\rm\thepage]{\fancyplain{}{\nouppercase{\sl{\leftmark}}}}
\rhead[\fancyplain{}{\nouppercase{\sl{\leftmark}}}]{\rm\thepage}
\pagestyle{fancy}

\begin{document}

\pagenumbering{roman}
\makeatletter

\newcommand{\HRule}{\rule{\linewidth}{0.2mm}}
%
\thispagestyle{empty}

\begin{center}\leavevmode

\vspace{-2cm}

\begin{tabular}{l}
\includegraphics[width=2.6cm]{template_tesis/logofcen.pdf}
\end{tabular}

{\large \sc Universidad de Buenos Aires}

{\large \sc Facultad de Ciencias Exactas y Naturales}

{\large \sc Departamento de Computaci\'on}

\vspace{4.0cm}

\begin{mdseries}
{\huge Inference of Socioeconomic Status}

{\huge in a Communications Graph:}

{\huge A Bayesian Approach}
\end{mdseries}

\vspace{2cm}

{\large Tesis presentada para optar al título de \\ Licenciado en Ciencias de la Computación}

\vspace{2cm}

{\Large \@author}

\end{center}

\vfill

{\large {Director: \director}}

\vspace{.2cm}

{\large {Codirector: \codirector}}

\vspace{.2cm}

{\large {\@date}}

\makeatother


\begin{abstract}
	%The explosion of mobile phone communications in the last years occurs at a moment where data processing power increases exponentially.  Thanks to those two changes in a global scale, the road has been opened to use mobile phone communications to generate inferences and characterizations of mobile phone users.
%In this work, we use the communications network, enriched by a set of users' attributes, to gain a better understanding of the demographic features of a population. Namely, we use
%call detail records and banking information to infer the income of each person in the graph.

%\todo{We need to rewrite the abstract.}

In this work, we examine the socio-economic correlations present among users in a mobile phone network in Mexico. First, we find that the distribution of income for a subset of users --for which we have income information given by a large bank in Mexico-- follows closely, but not exactly, the income distribution for the whole population of Mexico. We also show the existence of a strong socio-economic homophily in the mobile phone network, where users linked in the network are more likely to have similar income. The main contribution of this work is that we leverage this homophily in order to propose a methodology, based on Bayesian statistics, to infer the socio-economic status for a large subset of users in the network (for which we have no banking information). With our proposed algorithm, we achieve an accuracy of 0.71 in a two-class classification problem (low and high income) which significantly outperforms a simpler method based on a frequentist approach. Finally, we extend the two-class classification problem to multiple classes by using the Dirichlet distribution.

\end{abstract}

\prefacesection{Resumen}
% !TEX root = tesis.tex

Obtener y procesar datos demográficos y sociológicos fueron uno de los procesos más importantes para entender fenómenos que afectan a toda la población desde por lo menos el Siglo XVII~\cite{friendly2006}, y encontrar formas simples e intuitivas de visualizarlos tiene un gran impacto en nuestra manera de entender los datos~\cite{minard1844,snow1855}. Formas comunes de obtener datos cuantitativos de estratificación económica usualmente involucran investigación de archivos o encuestas sociales~\cite{bulmer1977},
% ; sin embargo esos métodos no pueden presentar datos que sean simultáneamente exactos, actualizados, y que aplican a una población grande sin depender de métodos estadísticos.
y dependen de métodos estadísticos.

Las operadoras de telecomunicaciones (``telcos'') tienen acceso a una gran cantidad de información sobre las comunicaciones y hábitos de sus usuarios~\cite{huurdeman2003}, pero la habilidad de guardar y procesar esos datos ha dado grandes pasos en los últimos años gracias a nuevas y más poderosas computadoras y técnicas de minería de datos. Lo mismo puede decirse sobre la información sociológica y económica contenida por bancos y tarjetas de crédito, y por la relación entre estas dos fuentes de datos.

La minería de datos de telcos a gran escala es un área relativamente nueva que se usa principalmente para aplicaciones internas~\cite{han2002emerging}, pero la gran cantidad de información sociológica es de gran interés para temas académicos relacionados a la sociología. Esta tesis se basa en métodos usaros por Óskarsdottir et.\ al.~\cite{oskarsdottir2016} y Singh et.\ al~\cite{singh2013predicting}, además de una fuente de información de una telco y de un banco grande para encontrar que la distribución de ingresos de los usuarios sigue de manera cercana (pero no exacta) la distribución de ingresos de la población en general.

Hay una fuerte homofilia entre los ingresos de contactos en la telco, que se usa junto con la distribución desigual de dinero en la población para crear una metodología, basada en estadística bayesiana, para inferir el nivel socioeconómico de un gran subconjunto de usuarios en la red sin información bancaria con $\AUC = 0.746$. El método bayesiano es luego comparado con otros métodos basados en aprendizaje automático supervisado para probar que, aunque toma menos información de entrada, es un mejor predictor de características sociales en este tipo particular de red.


\newpage
\frontmatter

\mainmatter

% !TEX root = tesis.tex

\chapter{Introduction}

% MOTIVATION
\section{Motivation of the Thesis}

In recent years, we have witnessed an exponential growth in the capacity to gather, store and manipulate massive amounts of data across a broad spectrum of disciplines: in astrophysics our capacity to gather and analyze massive datasets from astronomical observations has significantly transformed our capacity to model the dynamics of our cosmos; in sociology our capacity to track and study traits from individuals within a population of millions is allowing us to create social models at multiple scales, tracking individual and collective behavior both in space and time, with a granularity not even imagined twenty years ago.

In particular, mobile phone datasets provide a very rich view into the social interactions and the physical movements of large segments of a population. The voice calls and text messages exchanged between people, together with the call locations (recorded through cell tower usages), allow us to construct a rich social graph which can give us interesting insights on the users' social fabric, detailing not only particular social relationships and traits, but also regular patterns of behavior both in space and time, such as their daily and weekly mobility patterns~\cite{gonzalez2008understanding,ponieman2013human,sarraute2015city}.

Demographic factors play an important role in the constitution and preservation of social links. In particular concerning their age, individuals have a tendency to
establish links with others of similar age. This phenomenon is called age homophily~\cite{mcpherson2001birds}, and has been verified in mobile phone communications graph~\cite{blumenstock2010mobile,sarraute2014} as well as the Facebook graph~\cite{ugander2011anatomy}.


% PREVIOUS WORK ON THIS PROBLEM

Economic factors are also believed to have a determining role in both the social network's structure and dynamics. However, there are still very few large-scale quantitative analyses on the interplay between economic status of individuals and their social network. In~\cite{leo2015socioeconomic}, the authors analyze the correlations between mobile phone data and banking transaction information, revealing the existence of social stratification. They also show the presence of socioeconomic homophily among the networks participants using users' income, purchasing power and debt as indicators.
The authors of \cite{Luo2017inferring} studied the correlation between the position of a node in a mobile phone communications graph and its socio-economic status. They showed that the position and topological attributes in the graph can be used to generate inferences of the users' financial status.
In particular the study \cite{Luo2017inferring} shows the value of the Collective Influence~\cite{morone2015influence} as a topological attribute for the prediction of individual financial status.


% SUMMARY OF THE NEW APPROACH
\section{Summary of our Approach}

In this work, we leverage the socioeconomic homophily present in the cellular phone network to generate inferences of socioeconomic status in the communication graph. To this aim we will use the following data sources: (i) the Call Detail Records (CDRs) from the operator allow us to construct a social graph and to establish social affinities among users; (ii) banking reported income for a subset of their clients obtained from a large bank data source. We then construct an inferential algorithm that allows us to predict the socioeconomic status of users close to those for which we have banking information. To our knowledge, this is the first time both mobile phone and banking information has been integrated in this way to make inferences based on a social telecommunication graph.
Part of this work was published in~\cite{Fixman2016bayesian}.

\todo{Clearly state the hypothesis of the thesis}


Multiple strategies can be used to generate network features based on the CDRs. For instance, in \cite{oskarsdottir2016} the authors evaluate different collective inference methods applied to the churn prediction problem. Furthermore, the work \cite{oskarsdottir2017social} studies the impact of the social graph definition on the performance of the prediction methods. This motives the second part of the thesis, where we perform a comparative study of methods to generate network features for the nodes in the communication graph, and evaluate their impact on the inference of the income. We also compare the effectiveness of machine learning methods such as Logistic Regression and Random Forest on the different feature sets.

\todo{Add a summary of results}

\section{Organisation of the Thesis}

The remainder of the thesis is organized as follows.
In Chapter~\ref{chap:theoretical_intro} we provide an introduction to the theoretical ideas used in the thesis: the concept of homophily in social networks, \todo{complete}.

In Chapter~\ref{chap:related_work} we review related work on correlations in social-economic networks and on relation between socioeconomic status and mobile phone use. ETC \todo{complete}.

Chapter~\ref{sec:dataset} reviews the data sources used in this study. \todo{complete}.





% !TEX root = tesis.tex

% \chapter{Theoretical Introduction}
\chapter{Theoretical Building Blocks}
\label{chap:theoretical_intro}

\section{Social Homophily}

\epigraph{``People love those who are like themselves.''}{\textit{Rhetoric \\ Aristotle}}

Similarity breeds connection~\cite{mcpherson2001birds}. People have several visible characteristics, such as age, gender, and socioeconomic status, for which contact between people with similar properties occurs at a higher rate than between dissimilar people.

There are two overall types of homophily that can be distinguished in groups~\cite{lazarsfeld1954}: \textit{status homophily}, in which similarity is based on status, and \textit{value homophily}, which is based on values, attitudes, and beliefs. Status homophily, a part of which is the main study of this thesis, includes the major sociodemographic dimensions that stratify society --- ascribed characteristics like race, ethnicity, sex, or age, and acquired characteristics like religion, education, occupation, and behaviour patterns.

\subsection{Age Homophily}

One of the most common homophily patterns in human relations is related to the people's ages~\cite{ugander2011}\cite{mcpherson2001birds}. This result is expected because of the many societal reasons that explain the homophily: schools tend to group people according to age into the same classrooms, work opportunities tend to be clustered into age groups, which affects work environments and neighbourhood composition, and people have a strong tendency to confide in someone of one's own age.

This correlation has a waterfall effect. Since this kind of homophily is present since early into people's life, the produced connections are closer, longer lived, have a larger number of exchanges, and tend to be more personal than other kinds of connections.

There's an interesting exception to this homophily: there is a significant number of connections between parents and their younger children~\cite{sarraute2014}. This exception is addressed later in this paper.

\subsection{Gender Homophily}

McPherson et.\ al.\ also noted an important degree of homophily between members of the same gender~\cite{mcpherson2001birds}. In particular, ever since school age children learn that gender is a permanent personal characteristic, homophily can be observed in play patterns and friend groups.

By the time people are adults, people's friendship networks are relatively gender-integrated. However, when controlling for kinship networks and not counting close family members, there is a considerable level of gender homophily~\cite{marsden_1988}. However, this level is still lower than the one for race, education, age, and many other social dimensions.

Gender homophily is lower among the young and the highly educated\cite{marsden_1987}. One of the main reasons for this is that most environments where people make their networks, such as work establishments and voluntary organizations, are highly sex segregated. Therefore, it's not surprising that the networks formed in these settings display a significant amount of baseline homophily on gender.

\section{Spearman's Coefficient}
\label{subsec:spearman}

Spearman's Rank Correlation Coefficient (also known as Spearman's rho) is a non-parametric measure of rank correlation which measures how well the relationship between two variables can be described using a monotonic function~\cite{statistical_analysis}. Unlike Pearson's Correlation Coefficient, which measures lineal relationship between variables, Spearman's Coefficient uses the \emph{rank} of the variables in its calculations; therefore is measures its monotonicity.

For a sample of size $n$ with scores $X_i$ and $Y_i$, the Spearman Coefficient $r_s$ is defined as in \cref{eq:spearman}.

\begin{equation}
r_s = \mathlarger{\rho}_{\rank(X) \rank(Y)} = \frac{\cov(\rank(x), \rank(y))}{\sigma_{\rank(X)} \sigma_{\rank(Y)}}
\label{eq:spearman}
\end{equation}

Where $\rho_{a,b}$ denotes the \emph{Pearson Correlation} between the variables $a$ and $b$. This value will be closer to 1 when the variables are directly monotonic, closer to -1 when they are inversely monotonic, and closer to 0 when there is no tendency for either variable to increase or decrease when the other increases.

\section{Bayesian Inference}

\epigraph{``Given the number of times in which an unknown event has happened and failed: required the chance that the probability of its happening in a single trial lies somewhere between any two degrees of probability that can be named.''}{\textit{An Essay towards solving a Problem in the Doctrine of Changes~\cite{bayes1763} \\ Thomas Bayes}}

This work uses a Bayesian approach to statistics instead of the usual Frequentist approach. In the Frequentist point of view, parameters are fixed and unknown: hypotheses are either true or false, and they cannot be described with a probability. In the Bayesian approach, anything unknown is described with a probability distribution since uncertainty must be described by probability~\cite{mackay}.

\subsection{Bayes Theorem}

The base of \emph{Bayesian Inference} is \emph{Bayes' Theorem}, presented in \cref{eq:bayes}, which describes the probability of an event base on prior knowledge of conditions that may be related to it~\cite{gelman2003}.

\begin{equation}
\label{eq:bayes}
	P \left( H \mid E \right) = \frac{P \left( E \mid H \right) \cdot P \left( H \right)}{P \left( E \right)}
\end{equation}

Each one of the terms in \cref{eq:bayes} has a different definition and interpretation.

\begin{itemize}
	\item $P \left( H \mid E \right)$, the \textbf{Posterior Probability} is the conditional probability that is assigned after the relevant evidence is taken into account.
	\item $P \left( H \right)$, the \textbf{Prior Probability}, expresses the assumptions made on the problem before the experiments. While these assumptions will be subjective, the same thing can be said about the other probabilities in this model.
	\item $P \left( E \mid H \right)$, the \textbf{Likelihood}, is the degree of belief in $E$ given that $H$ is true. In most real world problems, this tends to be easier to define than the \emph{Prior}.
	\item $P \left( E \right)$, the \textbf{Marginal Likelihood}, as the likelihood function where some parameter variables were marginalized. It's used as a normalizing constant to that the \emph{Posterior Probability} integrates to 1, thus making it a valid probability. Since it's constant on the perspective of $H$, it's usually ignored when taking proportionality, as in \cref{eq:bayes_propto}.
\end{itemize}

It can be proven in a simple way by using basic theorems of the probability, as seen in \cref{eq:bayes_proof}.

\begin{equation}
\label{eq:bayes_proof}
\begin{aligned}
	P \left( H \cap E \right)
	&= P \left( H \mid E \right)P \left( E \right) \\
	&= P \left( E \mid H \right)P \left( H \right) \\
	P \left( H \mid E \right) P \left( E \right) &= P \left( E \mid H \right) P \left( H \right) \\
	P \left( H \mid E \right) &= \frac{P \left( E \mid H \right) P \left( H \right)}{P \left( E \right)}
\end{aligned}
\end{equation}

Most of the equations presented in this section deal with continuous probabilities, which by definition must integrate to 1~\cite{kolmogrov1956}. Therefore, the theorem is usually used in the version presented in \cref{eq:bayes_propto}, which defines the proportionality of the \emph{Posterior}.

\begin{equation}
\label{eq:bayes_propto}
	P \left( H \mid E \right) \propto P \left( E \mid H \right) \cdot P \left( H \right)
\end{equation}

\begin{figure}
\centering
\includegraphics[width=.7\textwidth]{figures/Bayes_Theorem.png}
\caption{Graphical visualization of \emph{Bayes Theorem} between two probabilities $A$ and $B$ by the superposition of two decision trees starting in hypothesis space $\Omega$.}
\label{fig:Bayes_Theorem}
\end{figure}

\subsection{Conjugate Priors}
\label{subsec:conjugate}

For a single problem there may be many different possible \emph{Prior Probabilities}, which can be defined depending on the approach taken on defining the model to represent different measures of knowledge and certainty about the data\footnotemark{}. In particular, if the prior is less informative then the posterior is more likely to be determined by the data.

\footnotetext{An extreme case is the \emph{Jeffreys Prior}, used to express total ignorance about the data~\cite{jeffreys453}.}

A simple way to choose a correct prior is using a \emph{Conjugate Prior}. A distribution $P \left( H \right)$ is \emph{Conjugate} to $P \left( H \mid E \right)$ if multiplying the two distributions together and normalizing the results in another distribution has the same form as $P \left( H \right)$.

The \emph{Conjugate Prior} has some philosophical significance in the context of \emph{Bayesian Estimator}. In the practical case, the \emph{Prior Probability} contains more or less information compared to the \emph{Posterior Probability} depending on the amount of data seen. In particular, if the experiment has seen little data, a single datapoint can influence your beliefs significantly. On the other hand, if the experiment has a lot of data, then one single extra datapoint shouldn't influence them as much~\cite{gelman2003}.

\section{The Beta Distribution}
\label{subsec:beta}

The \emph{Beta Distribution} is a family of continuous probability distributions defined in the interval $\left[ 0, 1 \right]$ which is parametrized by two shape parameters, $\alpha$ and $\beta$.

The distribution can be used to model the behaviour of \emph{Random Variables} limited to intervals of a finite length. It is often used as a statistical function to model unknown data from a known sample, such as allele frequencies in population genetics~\cite{Balding1995}, Malaysian sunshine data~\cite{Sulaiman1999573}, and heterogeneity in the probability of HIV transmission~\cite{SIM:SIM4780080110}.

In the context of \emph{Bayesian Inference}, the \emph{Beta Distribution} is the \emph{Conjugate Prior} of the \emph{Binomial Distribution}, which allows us to describe initial knowledge concerning probability of success of a single bi-variate distribution. In layman terms, this allows us to know what is the distribution of the continuous $p$ parameter of a binomial distribution for which we have $\alpha$ positive and $\beta$ negative samples.

\subsection{Probability Density Function}

Given a variable $0 \leq x \leq 1$, which represents the unknown probability of having a \emph{Positive Sample} from the distribution, and the shape parameters $\alpha > 0$ and $\beta > 0$, the \emph{Probability Density Function} of the beta distribution can be described as in \cref{eq:beta_pdf}, where $\kappa$ represents some constant.

\begin{equation}
\label{eq:beta_pdf}
\begin{aligned}
f\left(x; \alpha, \beta\right) &= \kappa \cdot x^{\alpha - 1} {\left( 1 - x \right)}^{\beta - 1} \\
&= \dsfrac{x^{\alpha - 1} {\displaystyle \left( 1 - x \right)}^{\beta - 1}}{\int^1_0 {u^{\alpha - 1} {\left( 1 - u \right)}^{\beta - 1} du}} \\
&= \dsfrac{\Gamma \left( \alpha + \beta \right)}{\Gamma \left( \alpha \right) \Gamma \left( \beta \right)} \cdot x^{\alpha - 1} {\left( 1 - x\right)}^{\beta - 1} \\
&= \dsfrac{1}{\Beta \left(\alpha, \beta\right)} \cdot x^{\alpha - 1} {\left(1 - x\right)}^{\beta - 1}
\end{aligned}
\end{equation}

\begin{equation}
\label{eq:beta_function}
\Beta\left(\alpha, \beta\right) = \frac{\Gamma\left(\alpha + \beta\right)}{\Gamma\left(\alpha\right) \Gamma\left(\beta\right)}
\end{equation}

\Cref{eq:beta_function}, describes the \emph{Beta Function}, which is related to the \emph{Gamma Function} and describes a similar pattern~\cite{thegammafunction}.

Regarding this thesis, the \emph{Beta Distribution} will be used to model a real life problem in \cref{sec:inference_methodology}. In this problem, both $\alpha \in \mathbb{N}$ and $\beta \in \mathbb{N}$, so the \emph{Beta Function} can be simplified using the identity $\left( x - 1 \right)! = \Gamma \left( x \right)$ as shown in \cref{eq:beta_int}.

\begin{equation}
\label{eq:beta_int}
\Beta(\alpha, \beta) = \frac{(\alpha + \beta - 1)!}{(\alpha - 1)! \cdot (\beta - 1)!}
\end{equation}

Additionally, the \emph{Beta Function} can be generalized into the \emph{Incomplete Beta Function} for some parameter $x$ as in \cref{eq:incomplete_beta}. This function is, confusingly, also represented with the Greek letter $\Beta$; to ease comprehension this thesis will refer to it as $\Betainc$.

\begin{equation}
\label{eq:incomplete_beta}
\Betainc(x; \alpha, \beta) = \int_0^x {t^{\alpha - 1} {(1 - t)}^{\beta - 1} dt}
\end{equation}

As we get more data from the sampling, the \emph{Beta distribution} turns more concentrated towards the actual $\theta$ and its shapes resembles more a normal curve, as can be seen in \cref{fig:betagraph}. This represents the increased certainty which comes from the acquired knowledge of the problem.

\begin{figure}
\centering
\includegraphicsmaybe{figures/beta.png}
\caption{Beta distribution with different parameters}
\label{fig:betagraph}
\end{figure}

\subsection{Cumulative Distribution Function}

The \emph{Cumulative Distribution Function} of the \emph{Beta Distribution} is defined in \cref{eq:beta_cdf_formula}.

\begin{gather}
\begin{gathered}
\label{eq:beta_cdf}
X \sim \Betadist \left( \alpha, \beta \right) \\
F \left( x; \alpha, \beta \right) = P \left( X \leq x \right)
\end{gathered} \\
\label{eq:beta_cdf_formula}
\begin{aligned}
F \left( x; \alpha, \beta \right)  &= \int_0^x f \left( t; \alpha, \beta \right) dt \\
&= \int_0^x {\frac{1}{\Beta \left( \alpha, \beta \right)} t^{\alpha - 1} {\left( 1 - t \right)}^{\beta - 1} dt} \\
&= \frac{1}{\Beta \left( \alpha, \beta \right)} \cdot \int_0^x {t^{\alpha - 1} {\left( 1 - t \right)}^{\beta - 1} dt} \\
&= \frac{\Betainc \left( x; \alpha, \beta \right)}{\Beta \left( \alpha, \beta \right)}
\end{aligned}
\end{gather}

$F$ is also known as the \emph{Regularized Incomplete Beta Function}, represented as $I_x(\alpha, \beta)$. This function is related to the \emph{Cumulative Distribution Function} of the \emph{Binomial Distribution}, as shown in \cref{eq:incomplete_beta_binomial}.

\begin{equation}
\label{eq:incomplete_beta_binomial}
\begin{gathered}
X \sim \Binom \left( n, p \right)  \\
P \left( X \leq k \right)  = I_{1 - p} \left( n - k, k + 1 \right)
\end{gathered}
\end{equation}

\subsection{Inverse Cumulative Distribution Function}
\label{subsec:beta_ppf}

The problems solved in this thesis require the use of the \emph{Inverse Cumulative Distribution Function} (also known as the \emph{Quantile Function} or the \emph{Percent-Point Function}) of the \emph{Beta Distribution}, which returns a value such $x$ that meets the expression in \cref{eq:beta_cdf} is equal to some value $p$. It can also be expressed as in \cref{eq:quantile_function}.

\begin{equation}
\label{eq:quantile_function}
Q \left( p \right)  = \inf \left\{ x \in \mathbb{R} \mid p \leq F \left( x \right) \right\}
\end{equation}

Like with the \emph{Cumulative Distribution Function}, there is no closed form formula for expressing its inverse~\cite{kippingexoplanets2013}. However, there are fast and accurate ways of computing it using either \emph{Interval~Halving} or \emph{Newton's~Method}, such as the \texttt{incbi} implementation in the \emph{Cephes} library~\cite{cephes} which is use in this thesis via a wrapper from \texttt{sklearn}, as explained in \cref{subsec:experimental_environment}.

\subsection{The Beta-Binomial Model}
\label{subsec:betabin}

In the \emph{Beta-Binomial Model} compromises a family of discrete probability distributions similar to the \emph{Binomial Distribution}, with the important difference that, instead of each trial having a constant probability of success, that probability is random and follows the \emph{Beta Distribution}~\cite{schervish1996statistics}.

Given a binary experiment which is run $n$ times, and the probability of success of any of those experiment is some constant $\theta$, the \emph{Probability Distribution} of the amount of successes $k$ can be modelled with a \emph{Binomial Distribution}, as shown in \cref{eq:betabin_binomial}.

\begin{equation}
\label{eq:betabin_binomial}
\begin{gathered}
	k \mid n, \theta \sim \Binomial \left( \theta, n \right) \\
	P \left( k = x \mid n, \theta \right) = \binom{n}{k} \cdot \theta^k {\left( 1 - \theta \right)}^{n - k}
\end{gathered}
\end{equation}

$\theta$ is a random continuous probability distribution, which is defined using the \emph{Beta Distribution} in \cref{eq:betabin_beta}.

\begin{equation}
\label{eq:betabin_beta}
\begin{gathered}
	\theta \mid \alpha, \beta \sim \Beta \left( \alpha, \beta \right) \\
	P \left( \theta \mid \alpha, \beta \right) = \frac{1}{\Beta \left( \alpha, \beta \right)} \cdot \theta^{\alpha - 1} {\left( 1 - \theta \right)}^{\beta - 1}
\end{gathered}
\end{equation}

Once the binary experiment is run, the model has additional information which may change the distribution of $\theta$. This can be modelled as a \emph{Posterior Distribution} using \emph{Bayes Theorem}~\cite{betabinomialcmu}.

\begin{equation}
\label{eq:betabin_bayes}
\begin{aligned}
	P \left( \theta \mid n, k, \alpha, \beta \right)
	&= \frac{P \left( k \mid n, \theta \right) P \left( \theta \mid n, \alpha, \beta \right)}{P \left( k \mid n, \alpha, \beta \right)} \\
	&\propto P \left( k \mid n, \theta \right) P \left( \theta \mid n, \alpha, \beta \right) \\
	&= P \left( k \mid n, \theta \right) P \left( \theta \mid \alpha, \beta \right) \\[1em]
	P \left( \theta \mid n, k, \alpha, \beta \right)
	&= \binom{n}{k} \theta^k {\left( 1 - \theta \right)}^{n - k} \cdot \frac{1}{\Beta \left( \alpha, \beta \right) } \theta^{\alpha - 1} {\left( 1 - \theta \right)}^{\beta - 1} \\
	&\propto \theta^k {\left( 1 - \theta \right)}^{n - k} \cdot \theta^{\alpha - 1} {\left( 1 - \theta \right)}^{\beta - 1} \\
	&= \theta^{k + \alpha - 1} {\left( 1 - \theta \right)}^{n - k + \beta - 1}
\end{aligned}
\end{equation}

This is exactly the same function as the one in \cref{eq:betabin_beta}. That is, the \emph{Posterior Distribution} of this model is also a \emph{Beta Distribution}.

\begin{equation}
\label{eq:betabin_posteriorbeta}
	\theta \mid k, n, \alpha, \beta \sim \Betadist \left( \alpha + k, \beta + n - k \right)
\end{equation}

This way it's possible to see that the \emph{Beta Distribution} has the properties of a \emph{Conjugate Prior Distribution} seen in \cref{subsec:conjugate} to the \emph{Binomial Likelihood}. This makes it extremely desirable for \emph{Bayesian Analysis}, and for this reason it's used as the main model of this thesis. This is seen in more detail \cref{subsec:modelling_users}.

\section{Machine Learning Validation Metrics}
\label{subsec:mlmetrics}

In the following subsections we present the outline of several supervised machine learning algorithms which are used to compare the Bayesian method to a more realistic baseline. First, we'll present several ways to validate the different algorithms when applied to the data. In the following section, we'll present many of the algorithms used for comparison.

Given a set of features $Z$, all of which belong to members of a population which belong to a certain category, and a random subset of those features $X \subseteq Z$ whose category $y$ is known, the models should be trained with $X$ and $y$ in order to correctly predict the values corresponding to all the features in $Z$. Since those values are unknown validation of the output is impossible; therefore, we validate the model using the known values in $X$ and $y$.

There are many metrics that can be used to measure the performance of a classifier or a predictor~\cite{binaryevaluation}; different fields have different preferences due to different goals. In this section, we present many metrics to evaluate different results that are commonly used in the area of mobile phone data analysis~\cite{oskarsdottir2016}.

\subsection{Classification of individual results}

Once we define out classifier $g$ and run it against a matrix of features, we get a predicted result $\ypred$ which, when compared to the actual result $\ytrue = y$, can be classified as the one in \cref{tab:confusion}.

\begin{table}
\begin{tabularx}{\textwidth}{| c | X | X X |}
\hline

& & \multicolumn{2}{c|}{\textbf{Predicted Condition}} \\
& Total Population &
\makecell{Condition Positive} &
\makecell{Condition Negative} \\ \hline

\multirow{2}{5em}{\textbf{True Condition}} &
Condition Positive &
\cellcolor{OrangeRed} \makecell{\textbf{True Positive}} &
\cellcolor{CadetBlue} \makecell{\textbf{False Negative} \\ (Type II error)} \\


& Condition Negative &
\cellcolor{CadetBlue} \makecell{\textbf{False Positive} \\ (Type I Error)} &
\cellcolor{OrangeRed} \makecell{\textbf{True Negative}} \\ \hline

\end{tabularx}
\caption[caption]{Confusion Table, showing different classifications of an individual prediction. True and False Positives ($\TP$/$\FP$) refer to the number of predicted positives that were correct/incorrect, and similarly for True and False Negatives ($\TN$/$\FN$).}
\label{tab:confusion}
\end{table}

Additionally, this table can be easily seen in a graphical way in \cref{fig:truefalsenegativepositive}.

\begin{figure}
\centering
\includegraphicsmaybe{figures/TrueFalseNegativePositive.png}
\caption{Visual explanation of \emph{Precision} and \emph{Recall}}
\label{fig:truefalsenegativepositive}
\end{figure}

\subsection{Precision and Recall}
\label{subsec:precisionrecall}
\emph{Precision} denotes the proportion of predicted positive cases that are correctly real positive. Trying to maximize this would allow us to adjust a particular predictor so that the majority of the predicted cases are actually positive. Conversely, \emph{recall} is the proportion of real positive cases that are correctly predicted positive, and maximizing it would allow us to adjust a predictor so that the majority of positive cases are predicted.

\begin{equation}
\begin{split}
\Precision = \TPA &= \frac{\TP}{\TP + \FP} \\
\Recall = \TPR &= \frac{\TP}{\TP + \FN}
\end{split}
\label{precisionrecall}
\end{equation}

These two measures and their combinations focus only on the positive examples and predictions, although between them they capture some information about the rates and kind of errors made~\cite{binaryevaluation}. While the \emph{recall} has been shown to have a major weight in working with machine translation~\cite{fraser2007}, they aren't particularly useful to use alone since they don't take into account many factors of the prediction~\cite{binaryevaluation}.

\subsection{Inverse~Precision and Inverse~Recall}

As a corollary of the previous metrics, we can add metrics that measure the proportion of real negative cases that are correctly predicted negative, referred as the \emph{Inverse~Recall}, and the proportion of predicted negatives that are real negatives, referred as the \emph{Inverse~Precision}\cite{binaryevaluation}. We can see that these are equivalent to finding the \emph{Precision} and \emph{Recall} of the negative category.

\begin{equation}
\begin{split}
\InvPrecision = \TNR &= \frac{\TN}{\FP + \TN} \\
\InvRecall = \TNA &= \frac{\TN}{\FN + \TN}
\end{split}
\label{negativeprecisionrecall}
\end{equation}

\subsection{Accuracy}
\label{subsec:accuracy}
The \emph{accuracy}, commonly referred in the context of binary classifiers as \textbf{Rand~Accuracy}\cite{powers15}, is used as a statistical measure of how well a binary classification test identifies or excludes a condition. Unlike the \emph{precision}, it takes into account the negatives, and it's expressible~\cite{binaryevaluation} both as a weighted average of \emph{precision} and inverse \emph{precision} or \emph{recall} and \emph{inverse recall}.

\begin{equation}
\Accuracy = \frac{\TP + \TN}{N}
\label{accuracy}
\end{equation}

This can be more simply expressed using the weighted average of either the \emph{Precision} and \emph{Inverse~Precision} or the \emph{Recall} and the \emph{Inverse~Recall}.

\begin{equation}
\begin{split}
\Accuracy &= \left(\TP + \TN\right) \cdot \TPR + \left(\FP + \TN\right) \cdot \TNR \\
&= \left(\TP + \FP\right) \cdot \TPA + \left(\FN + \TN\right) \cdot \TNA
\end{split}
\label{accuracy2}
\end{equation}

\subsection{ROC Curve}

A \emph{Receiver Operating Characterising} graph is a technique for visualizing, organizing, and selecting classifiers based on their performance~\cite{fawcett2005}. The curve is created by plotting the \emph{True Positive Rate} against the \emph{False Positive Rate} at various threshold settings.

This allows to compare different classifiers before having to select a particular threshold value for them. In particular, a random classifier will score near the positive diagonal ($\FPR = \TPR$), while a perfect classifier will score in the top left hand corner ($\FPR = 1, \TPR = 0$) and a worst case classifier will score in the bottom right hand corner\footnote{Note that, for any binary classifier, it's trivial to transpose the entire ROC curve (or a part of it) to the other part of the diagonal; therefore the worst ``realistic'' case is the random one}\cite{binaryevaluation}.

\begin{figure}
\centering
\includegraphics[width=.50\textwidth]{figures/ROC_example.png}
\caption{A \emph{ROC Curve}, where the \emph{Area Under the Curve} is marked. This particular graph comes from data used in early experiments to finding the socioeconomic index of a person, which is explained with more detail in \cref{sec:inference_methodology}.}
\label{fig:roc}
\end{figure}

\subsection{Area Under the Curve}
\label{subsec:auc}
The \emph{ROC Curve} allows us to compare classifiers and choose the one which is closer to optimal in some sense. While there are many possible parametrizations, the most common is to minimize the \emph{Area Under the Curve}, which is equal to the probability that a classifier will rank a randomly chosen positive instance higher than a randomly chosen negative one~\cite{fawcett2005}. This can be formulated as shown in \cref{eq:auc}.

\begin{equation}
\begin{aligned}
\AUC &= P\left(X_1 > X_0\right) \\
&= \int_0^1 \TPR \left( t \right) \FPR' \left( t \right) dt
\end{aligned}
\label{eq:auc}
\end{equation}

\subsection{F-measure}
\label{subsec:fmeasure}
The \emph{F-measure} is another measure of a tests accuracy. It considers both the \emph{Precision} and the \emph{Recall} of the test to compute the score. It can be considered the weighted average of both values for some weight $\beta$, where $F_\beta$ reaches the best score 1 when both precision and recall are 1.

\begin{equation}
\begin{split}
F_\beta &= \left( 1 + \beta^2 \right) \cdot \frac{\TPA \cdot \TPR}{\left( \beta^2 \cdot \TPA \right) + \TPR} \\
&= \frac{\left( 1 + \beta^2 \right) \cdot \TP}{\left( 1 + \beta^2 \right) \cdot \TP + \beta^2 \cdot \FN + \FP}
\end{split}
\end{equation}

The most commonly used \emph{F-measure}, $F_1$, measures the \emph{Precision} and \emph{Recall} is that harmonic mean of the \emph{Precision} and \emph{Recall}. In particular, for an \emph{F-measure} with $\beta > 1$ weights Recall higher than Precision, while with $\beta < 1$ weights Precision higher than Recall.

\subsection{Supervised Machine Learning Models}
\label{subsec:supervised_machine_learning}

This section presents several \emph{Supervised Machine Learning} models that are used in the paper. All models work using the process described in the introduction to Section~\ref{subsec:mlmetrics} with a 

\subsubsection{Logistic Regression}
\label{subsec:logisticregression}

\subsubsection{Decision Trees}
\label{subsec:decisiontrees}
\todo{Describe Decision Trees}

\subsubsection{Random Forest}
\label{subsec:randomforest}
\todo{Add Random Forest}




% !TEX root = tesis.tex

\chapter{Related Work}
\label{chap:related_work}


This thesis adds new data and experiments to the fast-growing area of \emph{Mobile Phone Social Network Analysis}.

Earlier works in the general area of \emph{Social Network Analysis} and \emph{Socioeconomic Indices} and their relation to demographic features were drawn from sparse sociological studies~\cite{katz_economics_2001} and nationwide surveys~\cite{deaton1997}. However, the advent of massive clusters of real-world data along with computers big enough to process it completely changed the landscape of human data analysis, both for industry purposes and for academia.

This chapter will discuss several scientific papers in this area which were relevant for the research done in this one.

\section{Correlations of Consumption Patterns in Social-Economic Networks}

\cite{leo16correlations} presents correlations between purchasing patterns and socioeconomic position of users from a dataset similar to the one used in this thesis. In particular, the authors have access to a database of credit card purchases for a set of users, with information about the amount of money spent and the general category (MCC) to which the purchase belongs, and also to a cellphone communications graph which allows them to infer the relationship between any two people.

The first of two interesting studies this paper makes is to categorize the population depending on their total spending, and find out the spending level of each user category on one of several aggregated purchase groups. It makes it easy to see the difference in spending for lower income and higher income people: the former group tends to spend comparatively more money in entertainment and retail stores, while the latter group spends more money in hotels and vehicles.

The second study presented in this paper relates to the correlation between people who buy from each of these groups to find categories which are commonly purchased together. Some groups, like \emph{Transportation}, \emph{IT}, or \emph{Personal Services} play a central role and are connected to many other communities, while some others like \emph{Car Sales and Maintenance} and \emph{Hardware Stores} and pairwise connected.

Both of these correlations can be seen intuitively in \cref{fig:paper_yannick}.


\begin{figure}
\centering
\begin{subfigure}[t]{.45\textwidth}
\includegraphicsmaybe{figures/yannick/service_socioeconomic.png}
\label{fig:service_socioeconomic}
\end{subfigure}
\begin{subfigure}[t]{.45\textwidth}
\includegraphicsmaybe{figures/yannick/service_service.png}
\label{fig:service_service}
\end{subfigure}
\caption{Data from the study of categories of purchases. The heatmap on the left side contains the relative purchasing quantities of each category for every socioeconomic level (where 1 is lower and 9 is higher), while the graph on the right side contains the purchase groups which reveals several clusters.}
\label{fig:paper_yannick}
\end{figure}

\section{Inferring Personal Economic Status from Social Network Location}

\cite{Luo2017inferring} shows that an individual's location is highly correlated with its socioeconomic status.
In addition, the paper also finds the interesting observation that some social network patterns mimic the economic inequality patterns, and that there is a significant ($R^2 = 0.96$) correlation between the link diversity of individuals and their financial status.

The results were validated by performing a social marketing campaign for the acquisition of new credit card clients by sending message for individuals that were predicted to be affluent.
Compared to a control group, the users with the most covariance between their links (that is, with the highest link diversity) would more probably request the offered product, which was ideal for affluent users. These results can be seen in \cref{fig:luo2017results}.

\begin{figure}
\centering
\includegraphics[height=.25\textheight]{figures/luo2017results.png}
\caption{Response rate versus \emph{Collective Influence}, a measure of the variance between links.}
\label{fig:luo2017results}
\end{figure}

Additionally, to prove that the results were not dependent on the validation campaign, the authors produced an \emph{Analysis of Covariance}\cite{wildt1978analysis} on all the features they had access to to test the variance caused by network metrics and other factors.
This resulted in the conclusion that the correlation between collective influence is positive and significant in all groups of geographical communities, across genders, and among all age groups older than 24 years.
Such robust network effects imply that network metrics are a potential indicator for financial status.

Unlike this thesis, this paper is completely observational and doesn't provide a direct inference method for socioeconomic status. However, both its strict methodology and its prediction of \emph{Collective Influence} were useful for completing many parts of this work, specially since the dataset used by Luo et.\ al has many similarities with the one used in this paper.


\section{Socioeconomic Status and Mobile Phone Use}

\cite{blumenstock2010mobile} combines data from direct demographic surveys with \emph{Call Details Records} obtained from a phone company to get demographical data about cellphone users in Rwanda.

The paper combines data about the overall demographic composition of Rwanda with the demographic composition of a representative sample of mobile phone users, along with voluntary survey results and the call history of the survey residents.

Two interesting tests made to measure the socioeconomic status of the respondents, which is particularly hard in a country where a significant percentage most people's income derives from informal channels.

\begin{itemize}
	\item Asking the respondents directly some of the demographic questions previously used in a nation-wide survey from the Rwandan government. This resulted a stark difference in socioeconomic level between the general population and the cellphone-owning people in the survey.
	\item Using this same government survey to compute total expenditures by aggregating expenditures across some subcategories as explained in~\cite{deaton2002}, and then fit the model to the data.
\end{itemize}

With this data it was possible to characterize economic stratification and inequality within the population of mobile phone users. Additionally, using the CDRs, it was possible to characterize graph properties for rich and poor users, in addition to other demographic indicators such as gender. In particular, while the mobile phone population is in general wealthier than the general population of Rwanda, there's still considerable inequality within the group of mobile phone users.


\section{Experimental Dataset}
\label{sec:dataset}

\subsection{Mobile Phone Data Source}
\label{subsec:mobiledatasource}

\subsubsection{Dataset Description}

The data used in this study consist of a multiset $P$ of composed of voice calls, and another multiset $S$ composed of text messages from a Mexican telecommunication company (\textit{telco}) for a 3 month period. These two sets are referred as the \emph{Call Detail Records}, or CDRs.

Every call $p \in P$ contains the phone numbers of the caller and callee $\left< p_o, p_d \right>$, which are anonymized using a cryptographic hash function for privacy reasons, the starting time $p_t$, and the call duration $p_s$. The same datum, except for the call duration, can be found for each element $s \in S$.

Additionally, the latitude and longitude of the antenna used for each call and SMS $\left< p_y, p_x \right>$  are given for certain users $V'$. Subsets $P' \subseteq P$ and $S' \subseteq S$ contain those calls.

Given that our collections $P$ and $S$ of CDRs are coming from one telephone company, we are able to reconstruct all communication links between clients of this company $N$, as well as communications between the clients and other users. However, we have no information on communications where neither users are clients of our telco company, and therefore users not in $N$ don't have complete call information.

The \emph{Communications Graph} $G = \left< V, E \right>$ is composed of he set of nodes $V = P_o \cup P_d \cup S_o \cup S_d$, and the set of directed edges $E$, where each element $e \in E$ is composed of an origin and destination $\left< e_o, e_d \right>$, the total amount of calls between these two users $e_c$, the total time of all calls $e_t$, and the amount of SMS $e_s$. Unlike the multisets $P$ and $S$, there is at most one element per every pair $\left< e_o, e_d \right>$ (althrough there may be two distinct elements with those values flipped).

The set $E$ can be formally constructed with the instructions of the Equation~\ref{eq:graphconstruction}.

\begin{equation}
\begin{gathered}
\label{eq:graphconstruction}
	\mathlarger{(\forall e \in E)} \\
	\left(
	\begin{aligned}
	e_c &= \left| \left\{ p \in P \mid \left< p_o, p_d \right> = \left< e_o, e_d \right> \right\} \right| \\
	e_s &= \left| \left\{ s \in S \mid \left< s_o, s_d \right> = \left< e_o, e_d \right> \right\} \right| \\
	e_t &= \sum_{\substack{p \in P \\ \left< p_o, p_d \right> = \left< e_o, e_d \right>}}{p_t} \\
	\end{aligned}
	\right)
\end{gathered}
\end{equation}

For simplicity sake, this paper will also refer to the elements of these three sets as $\calls_e$, $\sms_e$, and $\etime_e$ respectively.

\subsubsection{Magnitudes and Distributions}
\label{subsec:telco_magnitude}

As a corollary, $G_N$ can be defined as the graph $\left< N, E_N \right>$, where $E_N$ contains only calls between users of the telco, and $G'$ as $\left< V', E' \right>$, where $E'$ contains the calls from and to users whose calls are located.

The amount of each data contained in the dataset is described in Figure~\ref{tab:datasetnumbers}.

\begin{figure}
\centering
\begin{tabular}{>{\bfseries}l c c c}
\toprule
Dataset & Total Users & Telco Users & Located Users \\
Set & $G$ & $G_N$ & $G'$ \\
\midrule
Calls & \num{168287484} & \num{123456} & \num{123456} \\
SMS & \num{123456} & \num{123456} & \num{123456} \\
Time & \num{123456} & \num{123456} & \num{123456} \\
Contacts & \num{123456} & \num{123456} & \num{123456} \\
\bottomrule
\end{tabular}
\caption{Magnitudes of data about the different datasets. \todo{Complete this}}
\label{tab:datasetnumbers}
\end{figure}

Both the amount of calls and the amount of contacts per user are distributed logarithmically with a formula similar to Equation~\ref{eq:logdist}.

\begin{equation}
\label{eq:logdist}
	\left[ \left( \forall n \right) \frac{\left| \left\{ u \in G \mid \calls_u = n \right\} \right|}{\left| \left\{ u \in G \mid \calls_u = n + 1 \right\} \right|} \approx \kappa \right] \qquad \text{for some constant} \ \kappa
\end{equation}

This form of graph can be seen in Figure~\ref{fig:outcalls_dist} and Figure~\ref{fig:outcontacts_dist}.

\begin{figure}
\centering
\includegraphics[width=.75\textwidth]{figures/outcontacts_dist.png}
\caption{Distribution of amount of outgoing calls per user.}
\label{fig:outcalls_dist}
\end{figure}

\begin{figure}
\centering
\includegraphics[width=.75\textwidth]{figures/outcalls_dist.png}
\caption{Distribution of amount of outgoing calls per user.}
\label{fig:outcontacts_dist}
\end{figure}

Since, unlike with the amount of calls, the \emph{Average Call Time} isn't anywhere near 0, the pattern of \emph{Call Durations} follows a \emph{Negative Binomial Distribution}\maybe{Prove this} where a mean $\mu = 75$.

\begin{figure}
\centering
\includegraphics[width=.75\textwidth]{figures/callduration.png}
\caption{Distribution of the durations of calls.}
\label{fig:callduration}
\end{figure}

\subsection{Banking Information}
\label{subsec:bank_source}

\subsubsection{Dataset Description}

For this study we also obtained the set $B$ of account balanced of over 10 million clients of a bank in Mexico for a period of 6 months, which finishes at the same date as the period used in Section~\ref{subsec:mobiledatasource}. This dataset is represented by the set $B$, and each client $b \in B$ contains the phone number $b_p$, anonymized with the same hash as the datasets in the previous section, along with the reported income of this person in over 6 months $b_{s_0}, \ldots, b_{s_5}$. We average these 6 values to obtain $b_s$, the estimate of each users' monthly income.

The bank also provided us demographic information for a subset of its clients $A \subseteq B$. For each user $a \in A$, we are given the age $a_a$ and the gender $a_g$ of the user, which allows us to observe differences in the income distribution according to the age. The distribution is shown in Figure~\ref{fig:gender_age_bar}, 

\begin{figure}
\centering
\includegraphics[width=0.75\columnwidth]{figures/gender_age_bar3/gender_age_bar3.png}
\caption{Amount of users in $B$ by gender and age.}
\label{fig:gender_age_bar}
\end{figure}

The demographic data can be easily combined with the income data to show income by age, as figured in Figure~\ref{fig:income_age_boxplot}. The data shows how the median income increases with age up to the age of retirement, at around 60--65 years, and later it rapidly decreases.

In another line of work, homophily with respect to age has been observed and used to generate inferences~\cite{brea2014}.

\begin{figure}
\centering
\includegraphics[width=0.75\columnwidth]{figures/income_age_boxplot4/income_age_boxplot4.png}
\caption{Distribution of income $a_s$ as a factor of age $a_a$. This is consistent with data from median house income in Mexico~\cite{gallup2013}.}
\label{fig:income_age_boxplot}
\end{figure}

The income distribution, as shown in \Cref{fig:incomedistribution} presents a similar distribution to the values in the dataset presented in~\Cref{subsec:telco_magnitude}\maybe{Check this}.

\begin{figure}
\raggedright{}
\includegraphics[width=1.15\textwidth]{figures/incomedistribution.png}
% \includegraphicsmaybe{figures/incomedistribution.png}
\caption{Distribution of incomes of users, represented by the set $B$, with different percentiles marked. This plot helps appreciate the unequal distribution of income, since the 50th percentile has a comparatively low income, while the differential between higher percentile of incomes is each time higher.}
\label{fig:incomedistribution}
\end{figure}

\subsection{Matching of Bank and Telco Information}
\label{subsec:banktencomathing}

Since the phone numbers in each call in the list of users $V$ are anonymized with the same hash function as the phone number in the bank data in the set $B$, the users can be matched to their unique phone to augment the \emph{Social Graph} $G$, where the elements in the set $S = V \cap B$ contain banking information.

\begin{equation}
\label{eq:banktelcojoin}
\begin{gathered}
G = \left< V, E \right> \\
( \forall e \in E ) \\
e_o = b_p \implies e_{\operatorname{so}} = b_s \\
e_d = b_p \implies e_{\operatorname{sd}} = b_s \\
\end{gathered}
\end{equation}

The distribution of bank users in $S$ is similar to the one in $B$, as shown in~\Cref{fig:matchdistribution}. This demostrates that the users of this particular telco have mostly the same socioeconomic patterns as the clients of the bank in general.

\begin{figure}
\centering
\includegraphics[width=.75\textwidth]{figures/matchdistribution.png}
\caption{Distribution of incomes for users of both the bank and telco, represented by the set $S$.}
\label{fig:matchdistribution}
\end{figure}

\subsection{Outlier Filtering}
\label{subsec:outlier_filtering}

The dataset contains information about bank and telco users, some of which may not directly correspond to a human user, or may not have useful information for our research.

Most of the telco users in the first case are already filtered by the intersection between the bank and the telco data. However, to make sure the users are relevant enough for this study, we only keep the users which have a minimum amount of information, defined in the following items.

\begin{itemize}
	\item A monthly income of at least \$\num{1000}.
	\item A monthly income in the \num{99}th percentile (i.e.\ we filter users with a monthly income in the top 1\%).
\end{itemize}

\subsection{Unequal Distribution of Income}

We provide here some observations of the distribution of income of the bank clients. These observations correspond to the filtered dataset, obtained after applying the filters of the previous section.

\Cref{fig:lorenz} shows the Lorenz curve, graphical representation of the distribution of income~\cite{satchell1987}. The curve plots the cumulative share of clients, sorted by income, to the fraction of the total income of the population.

\begin{figure}
\centering
\includegraphics[width=0.75\columnwidth]{figures/cumulative_income.png}
\caption{Lorenz curve representing the distribution of income of bank clients.}
\label{fig:lorenz}
\end{figure}

From the Lorenz curve, we can compute the Gini coefficient as the area that lies between the line of perfect equality and the Lorenz curve over the total area under the line of equality. The data presents a coefficient of $\operatorname{Gini} = 0.45$.

According to the World Bank~\cite{world_bank}, the Gini coefficient for the whole population of Mexico was $0.481$ in 2012. Our result is consistent with this information, since the income inequality is expected to be lower when accounting only to bank clients than within the whole population of the country.

By simple analyzing \Cref{fig:incomedistribution}, we can reach the conclusion that thet income in Mexico is very unequally distributed. In particular, the top 1\% of the population get about 10\% of the total income, while the top 5\% get 25\%, the top 10\% 37\%, and the top 25\% get 61\%.

Analyzing \Cref{fig:incomedistribution}, we can observe that the top 10\% of the clients accumulate 33\% of the total income, while the top 20\% accumulate 50.5\% and the top 30\% accumulate 63.1\%; the rest of the income is distributed among the remaining 70\% of the population.

\subsection{Regional Distribution of Income}

Given the have locations of the calls in the subset $G' \subseteq G$, we can infer the home of each user in $V'$ using the method described in~\cite{csaji2013}. This is extremely useful when doing analyses, since it allows us to find differences in income distribution between areas of Mexico. In particular, \Cref{fig:regions} shows the average income per Mexican State in our dataset, which is compared to the data in Mexico in \Cref{tab:regions}.

\begin{figure}
\centering
\includegraphicsmaybe{figures/regions.png}
\caption{Average income by region of Mexico}
\label{fig:regions}
\end{figure}



\section{The Bayesian Method}
\label{inference_methodology}

%For the SEI:

%Census data + geolocalization + ARPU + cellphone payments (recharges) --> SEI estimation for geolocalized users.

%Second, we extended our predictions taking advantage of graph structure and socioeconomic homophily. To this end, we considered a bayesian

%\sout{As part of the dataset from the bank \( B \), we have the monthly salaries of most bank users \( B_{S_0} \cdots B_{S_t} \) for a period \( t \) larger than \( M \). We considered the average of \( B_{S_i} \) for a period of \( 6 \) moths to generate \( B_S \)}

%\sout{To infer the monthly salary of the users we take the average of \( B_{S_i} \) for a period of \( 6 \) moths to generate \( B_S \), and we compare them with other users' salaries by using the link correlations in \( G_N \).}

\subsection{Income Homophily}

\begin{figure}[h]
\begin{center}
{\includegraphics[width=\columnwidth] % ,trim={0.7cm 1.1cm 2.5cm 1.4cm},clip=true]
{figures/Homophily_income_origin_target_1/Homophily_income_origin_target_1.png}}
\caption{Heatmap showing the number of calls between users, according to their monthly income. There is a higher probability that the callee and the caller have similar income levels.}
\label{homophily_heatmap}
\end{center}
\end{figure}

The main contribution of this work is the estimation of the income of the telco users for which we lack banking data, but have bank clients in their neighborhood of the network graph. To show the feasibility of this task, we first show the existence of a strong income homophily in the telco graph as is evidenced in \Cref{homophily_heatmap}.

For each pair \( \left< o, d \right> \in \mathlarger{G} \), we define \( X \) as the set of incomes for callers and \( Y \) as the set of incomes for callees. According to what we can observe in \Cref{homophily_heatmap}, \( X \) and \( Y \) should be significantly correlated. Given the broad non-Gaussian distribution of the income's values, we choose to use a rank-based measure of correlation which is robust to outliers.

Namely we computed the \textit{Spearman's rank correlation} defined in \cref{spearman} to test the statistical dependence of sets \( X \) and \( Y \). This coefficient gives us a correlation coefficient of \( \mathbf{r_s = 0.474} \). We also compared our result with a randomized null hypothesis, where links between users are selected randomly disregarding income data, obtaining a \( p \)-value of \( p < 10^{-6} \). These values for \( r_s \) and \( p \) show a strong indication of income homophily among users in our communication graph. This observation is consistent with the results reported in~\cite{leo2015socioeconomic}.

We can take advantage of this homophily to propagate income information to the rest of our graph \( \mathP \), where we don't know the income of all the users.

\subsection{Prediction Algorithm}

Instead of predicting the exact value of a user's income, our strategy is to distinguish between only two income categories, \( R_1 = \closeopen{1000}{6300} \) and \( R_2 = \closeopen{6300}{\infty} \), that is, users with low or high income respectively, which we place into two distinct groups \( H_1, H_2 \subseteq G \) depending on \( g_s \), the users' income:
\[
	g \in H_i \iff g_s \in R_i
\]

We define the set \( Q \) as the group of users having at least one connection link to bank clients. For each user \( q^j \in Q \), we compute the number of outgoing calls \( a^j_i \) to the category \( H_i \). Our hypothesis, given the observed homophily, is that if a user \( q^j \) has a higher number of calls \( a^j_i \) to the category \( H_i \) than the other category, it would be more likely to belong to the \( H_i \) income category. In other words, a person is usually in the same income category as the majority of people it calls.

A straightforward approach would be to define the income category of a user as the category where most of its contacts belong. The problem with this approach is that it does not factor in the higher uncertainty in our estimates for users with fewer calls. To address this uncertainty, instead of using calling frequencies to define the probability of a user belonging to the high income category, we use the amount of calls \( a^j_i \) as parameters defining a Beta distribution for the probability of belonging to a given category. We have therefore taken a Bayesian rather than a frequentist approach to income prediction.

We define \(\Beta^j\) as the Beta probability distribution function for each user, which defines a distinct distribution for each user. Having obtained the Beta distribution for the probability of belonging to the high income category, we then find the lowest 5\textsuperscript{th} percentile \( p_{\operatorname{lower}} \) for this probability. If \( p_{\operatorname{lower}} \) is above a given threshold \( \tau \), we set the user's income to \( H_2 \), otherwise we set his income category to \( H_1 \). We note that this criteria takes into account both the mean and the broadness (uncertainty) of the distribution. We also note that the category assigned to a user depends not only on its Beta distribution but also on our choice of \( \tau \).

%We therefore choose a value for \( \tau \) which maximizes the trade off between true positive (\( TPR \)) and false positive (\( FPR \)) rates: \( TPR=TP/P \) and \( FPR=FP/N \) where \( TP \) is the number of correctly predicted users with high income, \( P \) is the total number of users with high income, \( FP \) is the number of users incorrectly classified as having high income, and \( N \) is the total number of users with low income.

%For each user \( q^j \) we estimate and the corresIn this way we obtained a Dirichlet distribution for each user \( q^j \) and compute

%According to the Dirichlet Distibution, each user has a caller income category probability density function of the form

%For all the other users in the telco having at least one link to any other telco user with a defined income \( q \in Q = \left\{x \in \left( N \setminus B \right) \mid (\exists y \in G) \left< x, y \right> \in P \vee \left< y, x \right> \in P \right\} \), we can use this distribution to infer the probabilities of being part of each group \( H_1, \ldots, H_5 \), and this way approximate the economic status.


\section{Results}

% \subsection{Validation Process}

%We need to make sure that our inference algorithm correctly predicts users' income in each of the 5 categories $ H_1, \ldots, H_5 $. According to our hypothesis, the greater the amount of calls to , this increases the probability that both of them will belong to the same group.

%For each link $ p = \left< p_o, p_d, \ldots \right> \in P $, our hypothesis says that $ p_o $ and $ p_d $ will belong to the same income group. 

%In order to validate our proposed methodology, we take advantage of the computed $p_lower$ for each user and use this quantity as the predicition score that allow us to evaluate whether a given user belongs to one of the deffined categories. A standarized way to quantify the performace of the metodology is by computing Receiver Operating Characteristic Curve (ROC). This curve  

We describe in this section the validation of our methodology. 
We examine the true positive ($TPR$) and false positive ($FPR$) rates: 
$$TPR=TP/P \; \; \mbox{ and } \; \;  FPR=FP/N$$ 
where $TP$ is the number of correctly predicted users with high income, $P$ is the total number of users with high income, $FP$ is the number of users incorrectly classified as having high income, and $N$ is the total number of users with low income. 

In Figure~\ref{ROC_multiclass} we plot the ROC (\textit{Receiver Operating Characteristic}) curve, showing $TPR$ and $FPR$ for the set of possible values of $\tau$. We see that our methodology clearly outperforms random guessing (dashed straight line). We can summarize our performance by calculating  the AUC (\textit{Area Under the Curve}) which in Figure~\ref{ROC_multiclass} is $AUC = 0.74$. Note that random guessing would give a value of $AUC \simeq 0.50$.


%\vspace{1em}
%
%\begin{adjustwidth}{-1.5cm}{}
%\begin{tabu} to \textwidth { r X[c] X[c] }
%& \multicolumn{2}{ c }{\textbf{Classifier result for income group $ H_i $}} \\
%\\
%& Caller in group & Caller not in group \\
%Callee in group & \cellcolor{green} True Positive & \cellcolor{red} \makecell{False Negative \\ (Type II Error)} \\ 
%Callee not in group & \cellcolor{red} \makecell{False Positive \\ (Type I Error)} & \cellcolor{green} True Negative \\
%\end{tabu}
%\end{adjustwidth}

%To validate these classifiers, we use 

%a \textbf{Receiver Operating Characteristic Curve} (ROC) to plot the \textbf{True Positive Rate} and the \textbf{False Positive Rate} of each system. Since the amount of people in each group is almost equal, randonly guessing each user's income would result in a diagonal line.

%As an objective measurement, we can calculate the \textbf{Area under the curve} for each income group as the chance that the classifier would make a better prediction of the composition of this income group better than a random one. For this we define the density functions $ f_i(x) $ as the probability that a user will be considered part of the income group $ H_i $, and $ g_i(x) $ as the probability that it won't. This way we can characterize the True Positive Rate and the False Positive Rate for each group as

%\vspace{-1em}
%
%\begin{align*}
%\operatorname{TPR}_i(T) &= \int^{\infty}_T f_i(x) dx \\
%\operatorname{FPR}_i(T) &= \int^{\infty}_T g_i(x) dx \\
%\end{align*}
%
%\vspace{-1.5em}

%With this two functions, we can define the AUC for each category as as a simple integration:

%\[
%A_i = P(X_1 > X_0) = \int^{-\infty}_{\infty} \operatorname{TPR_i}(T) \operatorname{FPR_i}(T) dT
%\]

%Where $ X_1 $ is the score for a positive instance, and $ X_0 $ is the score for a negative one.

%\subsection{Results}

\begin{figure}[H]
\begin{center}
\includegraphics[width=\columnwidth]{figures/ROC_BETA/ROC_Beta_based_approach_201504.png}
\caption{ROC curve for prediction procedure. We observed an $AUC = 0.74$ indicating that our predictor is better than a random predictor ($AUC \simeq 0.50$).}
\label{ROC_multiclass}
\end{center}
\end{figure}

%The result ROC curves have the following areas:
%
%\vspace{-1em}
%
%\begin{align*}
%A_1 &= 0.68 & A_2 &= 0.69 & A_3 &= 0.63 \\ 
%A_4 &= 0.68 & A_5 &= 0.69
%\end{align*}
%
%As evidenced by figure \ref{ROC_multiclass}, the inference process seems to have been successful: the predicted value is much better than one predicted by a random classifier for all of the categories.


\section{Comparison with Other Inference Methods}
\label{sec:comparison}

In the following section, we compare the results of the Bayesian method with other classifiers commonly used for this same problem.

This classified creates a proper baseline for other comparisons. These will be used in the dataset $G = \left< V, E \right>$ with the bank data contained in $B^{\train}$, and will be used eihter to provite information about the users contained in $B^{\test}$ (the \emph{Outer Graph}), or the ones in $\Upsilon$ (the \emph{Inner Graph}). Those two sets of features are described with more detail in Section~\ref{subsec:train_test_split} and Section~\ref{subsec:rebalancing_labels}.

For simplicity sake, the descriptions of the algorithm will refer to the \emph{Testing Set} as $\Upsilon$ regardless of whether it's using a subset or the data or not.

Table~\ref{tab:comparison} compares the results of these methods with the \emph{Bayesian Algorithm} described in Section~\ref{sec:inference_methodology} along several metrics.

\subsection{Random Selection}
\label{subsec:random_selection}

For completeness sake we created a random ``dummy'' classifier which randomly chooses the socioeconomic index of each user.

\begin{equation}
\label{eq:random}
\begin{aligned}
	P \left( v \in H_1 \right) &= \sfrac{1}{2} \\
	P \left( v \in H_2 \right) &= \sfrac{1}{2}
\end{aligned}
\end{equation}

\subsection{Majority Voting}
\label{subsec:majority_voting}

The method of \emph{Majory Voting} is a basic but powerful way of inferring to which category a user belongs. It simply chooses the category of each user $\upsilon \in \Upsilon$ as the category for which the majority of its contacts belong.

In case of a tie (which happens often when using the \emph{Outer Graph}), the category is chosen randomly.

\begin{equation}
\label{eq:majority_voting}
\begin{aligned}
	\contacts^{\low}_{\upsilon} &> \contacts^{\high}_{\upsilon} \implies \upsilon \in H_1 \\
	\contacts^{\low}_{\upsilon} &< \contacts^{\high}_{\upsilon} \implies \upsilon \in H_2 \\
	\contacts^{\low}_{\upsilon} &= \contacts^{\high}_{\upsilon} \implies
	\begin{cases}
		\upsilon \in H_1 \  \text{with probability} \ \sfrac{1}{2} \\
		\upsilon \in H_2 \  \text{with probability} \ \sfrac{1}{2}
	\end{cases}
\end{aligned}
\end{equation}

\subsection{Averaging Contacts' Incomes}

\todo{Complete this}

\subsection{Methods Based in Machine Learning}
\label{subsec:methods_ml}

The following methods use commonly used supervised Machine Learning algorithms described in Section~\ref{subsec:supervised_machine_learning} used with many similar \emph{Feature Extraction} methods on $G$.

Each method is identified with an integer $n \in \mathbb{N}$ which represents the size of the \emph{Ego Network} used to get information for each node. Additionally, some categories have \emph{Category Data} from $B^{\train}$ to help identify the network.

Additionally, the features of each method are merged with the ones from the immediate predecessor methods, as shown in the graph in Figure~\ref{fig:mlrelationships}.

\begin{figure}
\centering
\includegraphicsmaybe{figures/mlrelationships.png}
\caption{Relationships between the \emph{Feature Extraction} methods of Section~\ref{subsec:methods_ml}}
\label{fig:mlrelationships}
\end{figure}

\subsection{User Data --- Level 0}
\label{subsec:user_data}

The features on the graph $G = \left< V, E \right>$ can be accumulated for all users in a manner similar to the one in Section~\ref{subsec:feature_accumulation} using the data generated in Equation~\ref{eq:graphconstruction}. However, this time we aren't constrained either by having only users which have contacts with other users with known \emph{Income Category}, nor with having to have only two features for each category (which was necessary due to using the \emph{Beta Distribution}).

This way, it's possible to accumulate features for each user $v \in V$ in the manner shown by Equation~\ref{eq:user_data}.

\begin{equation}
\label{eq:user_data}
\begin{gathered}
\begin{aligned}
\incalls_v &= \sum_{\substack{e \in E \\ e_d = v}} \calls_e &
\outcalls_v &= \sum_{\substack{e \in E \\ e_o = v}} \calls_e \\
\intime_v &= \sum_{\substack{e \in E \\ e_d = v}} \etime_e &
\outtime_v &= \sum_{\substack{e \in E \\ e_o = v}} \etime_e \\
\insms_v &= \sum_{\substack{e \in E \\ e_d = v}} \sms_e &
\outsms_v &= \sum_{\substack{e \in E \\ e_o = v}} \sms_e \\
\end{aligned} \\
\begin{aligned}
\incontacts_v &= \left| \left\{ e \in E \mid e_d = v \right\} \right| \\
\outcontacts_v &= \left| \left\{ e \in E \mid e_o = v \right\} \right|
\end{aligned}
\end{gathered}
\end{equation}

These features will be referred as the \emph{User Data} of user $v$.

There features contain information about the \emph{Neighbourhood} of $\upsilon$, also referred to as the \emph{Ego Network of Distance 1}. This \emph{Neighbourhood} can be formally defined as in Equation~\ref{eq:neighbourhood}.

\begin{equation}
\label{eq:neighbourhood}
\neigh \left( \upsilon \right) = \left\{ e_o \mid e \in E \and e_d = \upsilon \right\} \cup \left\{ e_d \mid e \in E \and e_o = \upsilon \right\}
\end{equation}

\subsection{Categorical User Data --- Level 0'}
\label{subsec:categoricaluserdata}

A possible featureset consists of using of separating the data of the neighbourhood of each user $\upsilon \in \Upsilon$ into two disjoint groups, $L_{\upsilon}$ and $K_{\upsilon}$, which contain the neighours of $\upsilon$ in the \emph{Low Income} and \emph{High Income} categories of income respectively\footnotemark{}.

\footnotetext{Note that, since not all users have banking information, there may be nodes in the neighbourhood of $\upsilon$ which don't belong to either $L_{\upsilon}$ or $K_{\upsilon}$.}

\begin{equation}
\label{eq:cud_categories}
\begin{aligned}
	L_{\upsilon} &= H_1 \cap \neigh \left( \upsilon \right) \\
	K_{\upsilon} &= H_2 \cap \neigh \left( \upsilon \right)
\end{aligned}
\end{equation}

Having these groups it's possible to define a set of features similar to the one in Section~\ref{subsec:user_data}, where each feature is separated by the category of the neighbour. Equation~\ref{eq:matcatuserdata} contains the names of the new features.

\begin{equation}
\label{eq:matcatuserdata}
	\begin{Bmatrix} in \\ out \end{Bmatrix}
	\times
	\begin{Bmatrix} calls \\ time \\ sms \\ contacts \end{Bmatrix}
	\times
	\begin{Bmatrix} low \\ high \end{Bmatrix}
\end{equation}

Equations~\ref{eq:categoricaluserdata} and~\ref{eq:categoricaluserdata2} contain the way to calculate those features. Since the number of individual features is high and the formulas are similar and repetitive, some generalizations were added for the simplest ones.

\begin{equation}
\label{eq:categoricaluserdata}
\begin{gathered}
	\left( \forall \varpi \in \left\{ \calls, \etime, \sms \right\} \right) \\
\begin{aligned}
	\underline{\text{in}{\varpi}\text{low}}_v = \sum_{\substack{e \in E \\ e_d \in L_v \\ e_o = v}} &\varpi_e &
	\underline{\text{in}{\varpi}\text{high}}_v = \sum_{\substack{e \in E \\ e_d \in K_v \\ e_o = v}} &\varpi_e \\
	\underline{\text{out}{\varpi}\text{low}}_v = \sum_{\substack{e \in E \\ e_o \in L_v \\ e_d = v}} &\varpi_e &
	\underline{\text{out}{\varpi}\text{high}}_v = \sum_{\substack{e \in E \\ e_o \in K_v \\ e_d = v}} &\varpi_e \\
\end{aligned}
\end{gathered}
\end{equation}

\begin{equation}
\label{eq:categoricaluserdata2}
\begin{aligned}
	\incontactslow_v   &= \left| \left\{ e \in E \mid e_d = v \land e_o \in L_v \right\} \right| \\
	\incontactshigh_v  &= \left| \left\{ e \in E \mid e_d = v \land e_o \in K_v \right\} \right| \\
	\outcontactslow_v  &= \left| \left\{ e \in E \mid e_o = v \land e_d \in L_v \right\} \right| \\
	\outcontactshigh_v &= \left| \left\{ e \in E \mid e_o = v \land e_d \in K_v \right\} \right|
\end{aligned}
\end{equation}

Unlike the features in Section~\ref{subsec:user_data}, and like the method presented in Section~\ref{sec:inference_methodology}, the features in this section will be be different when testing between the nodes of $B_{\test}$ (the \emph{Outer Graph}) and $\Upsilon$ (the \emph{Inner Graph}).

\subsection{Higher Order User Data}
\label{subsec:higherorderuserdata}

Where \emph{User Data} contains data directly about the neighbours of each user, we can define \emph{User Data of Order $1$}, as the user data variables about the edges of the neighbours of each user $u$ where $u$ isn't any of the endpoints. In the case of cumulative data (calls, time, and SMS), this is equal to the sum of the \emph{User Data} of the neighbours of $u$ minus the \emph{User Data} os $u$ (since we aren't counting ``inner'' edges). This isn't true in the case of the contacts, since two neighbours of $u$ may have a contact in common, which should be counted only once.

Additionally, for any $n \in \mathbb{N}$, we can inductively define the \emph{User Data of Order $n$} as the user data of the nodes at distance $n$ of a certain node.

\subsection{Validation Metrics}
\label{subsec:validationmetrics}
There are several validation metrics used for each method.

\begin{description}
	\item[Accuracy] as described in Subsection~\ref{subsec:accuracy}, which measures the general performance of this method.
	\item[Precision] as described in Subsection~\ref{subsec:precisionrecall}, which measures the performance regarding the positive instances found by this method.
	\item[Recall] as described in Subsection~\ref{subsec:precisionrecall}, which measures the performance regarding the positive instances in the dataset.
	\item[Area Under the Curve] as described in Subsection~\ref{subsec:auc}, which measures the general performance disregarding which threshold is used.
	\item[$\mathbf{F_1}$ Score] as described in Subsection~\ref{subsec:fmeasure} which is generalized score balancing Precision and Recall.
	\item[$\mathbf{F_4}$ Score] as described in Subsection~\ref{subsec:fmeasure}, which gives more weight to the Recall. This is usually wanted since the ultimate practical objective of this study is to find wealthier people, even if the result has low Precision.
	\item[Time] can be used to break ties between similar models.
\end{description}

\subsection{Feature Extraction}
\label{subsec:featureextraction}
Using the previous graph, we can create features separated into different levels.

\begin{enumerate}
	\item[0] \emph{Local Features} as described in Section~\ref{subsec:user_data}.
	\item[0.5] \emph{Classified Local Features} as described in Section~\ref{subsec:categoricaluserdata} (using only the adjacency information of the training set $T$), joined with the data on the previous item. There are two separate sets of sampels used for experiments on this level.
	\begin{description}
		\item[Outer Nodes] which contain all samples in $S$.
		\item[Innter Nodes] which only contain samples in the testing set $S$ which have at least a neighbour in the training set $T$.
	\end{description}
	\item[1] \emph{Neighbour Features}, as described in Section~\ref{subsec:higherorderuserdata} for \emph{User Data of Order 1}, joined with data on \textbf{Level 0}.
	\item[1.5] \emph{Classified Neighbouring Features}, which combine the feature extraction methods seen in Section~\ref{subsec:categoricaluserdata} and Section~\ref{subsec:higherorderuserdata}, and separate the accumulation of the \emph{User Data} of a node's neighbours along categorical features. This data is also joined with all previous levels, and discriminated between sets of \emph{Outer Nodes} and \emph{Inner Nodes}
	\item[$\geq 2$] The method described in Section~\ref{subsec:higherorderuserdata}, with or without classification by the type of the other endpoint of the link. In this section, we'll prove that the accuracy of the methods before this one are similar enough to the ones in this level so that sacrificing speed is not worth the new features.
\end{enumerate}

\begin{table}
\centering
\begin{tabular}{>{\bfseries}l c c c c c c c}
\toprule
\textbf{Level} & 0 & \multicolumn{2}{c}{0.5} & 1 & \multicolumn{2}{c}{1.5} & 2 \\
\cmidrule(lr){3-4} \cmidrule(lr){6-7}
\textbf{Dataset} &   & inner     &    outer    &   & inner     &    outer    &   \\
\midrule
Features &\num{1234567}&\num{1234567}&\num{1234567}&\num{1234567}&\num{1234567}&\num{1234567}&\num{1234567}\\
Samples  &\num{1234567}&\num{1234567}&\num{1234567}&\num{1234567}&\num{1234567}&\num{1234567}&\num{1234567}\\
\bottomrule
\end{tabular}
\caption{Size of the datasets used for testing for each feature extraction method}
\label{tab:datasettable}
\todo{Change placeholders for actual numbers}
\end{table}

These datasets contain differing numbers of features and samples, as described in Table~\ref{tab:datasettable}.

\subsection{Results}

Applying the features from Section~\ref{subsec:featureextraction} to the \emph{Random Forest} model described in Section~\ref{subsec:randomforest} and the \emph{Logistic Regression} model described in Section~\ref{subsec:logisticregression}. Prior to applying each model, a \emph{Grid Search} was made with the parameter $C$ in the case of the \emph{Logistic Regression} as shown by Equation~\ref{eq:gridsearch}, and the \emph{Criterion} used in the \emph{Random Forest}, and the hyperparameter which results in the most \emph{Accuracy} after 5-fold cross validation was used.

\begin{equation}
\label{eq:gridsearch}
\begin{split}
C &\in \left\{ 0.01, 0.1, 1, 10, 100 \right\} \\
\operatorname{Criterion} &\in \left\{ \text{Gini}, \text{Entropy} \right\}
\end{split}
\end{equation}

The results of the \emph{Grid Search} are presented in Table~\ref{tab:gridsearch}, while the different metrics of the result are presented in Table~\ref{tab:comparison}.

\begin{table}
\centering
\begin{tabular}{>{\bfseries}l l @{\hskip 2em} r r}
\toprule
Level & Dataset & $C$ (LR) & Criterion (RF) \\
\midrule

0 & & \num{10} & Entropy \\ [1.5ex]

\multirow{2}{*}{0.5} & Outer & \num{0.01} & Gini \\
& Inner & \num{1} & Gini \\ [1.5ex]

1 & & \num{100} & Entropy \\ [1.5ex]

\multirow{2}{*}{1.5} & Outer & \num{100} & Entropy \\
& Inner & \num{100} & Entropy \\
\bottomrule

\end{tabular}
\caption{Best hyperparameters for each group of features in each model used}
\label{tab:gridsearch}
\end{table}

\begin{table}
\begin{tabular*}{\textwidth}{>{\bfseries}l l l @{\extracolsep{\fill}} r r r r r r r}
\toprule
Level & Dataset & Method & Accuracy & Precision & Recall & AUC & F\textsubscript{1}-score & F\textsubscript{4}-score & Time \\
\midrule

\multirow{2}{*}{0}
& & LR & \num{0.541} & \num{0.580} & \num{0.299} & \num{0.541} & \num{0.395} & \num{0.308} & \SI{2.647}{\second} \\
& & RF & \num{0.541} & \num{0.542} & \num{0.528} & \num{0.541} & \num{0.535} & \num{0.528} & \SI{11.396}{\second} \\
\midrule

\multirow{4}{*}{0.5}
& \multirow{2}{*}{Outer} & LR &\num{0.570} & \num{0.749} & \num{0.212} & \num{0.570} & \num{0.330} & \num{0.221} & \SI{3.727}{\second} \\
& & RF &\num{0.567} & \num{0.572} & \num{0.533} & \num{0.567} & \num{0.552} & \num{0.535} & \SI{8.528}{\second} \\
\cmidrule{2-10}
& \multirow{2}{*}{Inner} & LR &\num{0.720} & \num{0.747} & \num{0.842} & \num{0.675} & \num{0.792} & \num{0.836} & \SI{1.005}{\second} \\
& & RF &\num{0.700} & \num{0.738} & \num{0.817} & \num{0.658} & \num{0.775} & \num{0.812} & \SI{2.146}{\second} \\
\midrule

\multirow{2}{*}{1}
& & LR & \num{0.552} & \num{0.597} & \num{0.326} & \num{0.553} & \num{0.422} & \num{0.335} & \SI{5.255}{\second} \\
& & RF & \num{0.572} & \num{0.580} & \num{0.527} & \num{0.572} & \num{0.552} & \num{0.530} & \SI{22.523}{\second} \\
\midrule

\multirow{4}{*}{1.5}
& \multirow{2}{*}{Outer} & LR & \num{0.598} & \num{0.726} & \num{0.358} & \num{0.606} & \num{0.479} & \num{0.369} & \SI{22.851}{\second} \\
& & RF & \num{0.723} & \num{0.747} & \num{0.852} & \num{0.675} & \num{0.796} & \num{0.845} & \SI{4.780}{\second} \\
\cmidrule{2-10}
& \multirow{2}{*}{Inner} & LR & \num{0.643} & \num{0.668} & \num{0.617} & \num{0.644} & \num{0.642} & \num{0.620} & \SI{21.498}{\second} \\
& & RF & \num{0.735} & \num{0.766} & \num{0.838} & \num{0.697} & \num{0.800} & \num{0.834} & \SI{6.706}{\second} \\
\midrule

\multirow{2}{*}{2}
& & LR & \num{0} & \num{0} & \num{0} & \num{0} & \num{0} & \num{0} & \SI{0}{\second} \\
& & RF & \num{0} & \num{0} & \num{0} & \num{0} & \num{0} & \num{0} & \SI{0}{\second} \\
\bottomrule

\end{tabular*}
\caption{Results of running dataset with different feature extraction methods}
\label{tab:comparison}
\todo{Add level $\geq 2$}
\end{table}


\section{Conclusion}

This work is based on the combination of two data sources of mobile phone records and banking information. We showed that there is a significant level of homophily between the income of the participants of a call, and based on this property, we presented a Bayesian approach to infer the income category of users in the graph for which we don't have banking data.

We first classified users into 2 categories depending on their income. To this end, we computed the number of calls each user \( u \) makes to members of the same and different categories, and we constructed a Beta distribution for the probability of user \( u \) belonging to each category. We later validated this approach by constructing the ROC curve and computing its accuracy, and compared it to random guessing and to a simpler method based on majority voting. We were able to validate that the method presented outperforms the other two.

Our proposed inference methodology is useful in concrete applications, since it provides an estimation of socio-economic attributes of users lacking banking history, based on their communication network. We also note that this methodology is not restricted to the inference of socio-economic attributes, but is equally applicable to any attribute that exhibits significant homophily in the network.

%\todo{Revisar conclusions}
%
%\todo{Para una proxima version:}
%
%\todo{Ampliar a vecinos de vecinos}
%
%\todo{Comparar con otros metodos}
%
%\todo{Comparar con reacción-difusión (a primeros vecinos)}
%
%\todo{Idea de Alejo: usar los parametros de la Dirichlet (usar un prior no uniforme)}
%
%\todo{Distribucion de income inferido en función de la edad}


\bibliography{bibliography/sna}{}

\end{document}
