\chapter{Conclusions}

\section{General Objectives}

This thesis presented several methods of manipulating communication and socioeconomic data to produce useful insights about its users.
The combination of different datasets present an unique opportunity to study group behaviour and to experiment with gathering data that isn't directly present as an input.

The first data source used in this study is the multiset of mobile phone \emph{Call Detail Records} $P$ (for phone calls) and $S$ (for SMS), which is presented in \cref{subsec:dataset_description}.
Such a complete datasets of all the communication done by the users of a certain telco, which also includes contact to and from users from outside this particular company, allowed us to create a \emph{Social Graph} which allows us to find many insights about its members.

The second data source is the set with \emph{Bank Information} about the users $B$, which is presnetd in \cref{subsec:bank_source}.
While the values in this set aren't perfectly correlated to the socioeconomic index of the users, its data is a good enough proxy for the object of this study.
For this reason, we separated the users into 2 distinct groups\footnotemark{}.

\footnotetext{The input data, including the source county, is obfuscated. The symbol \$ refers to a certain currency that isn't necessarily the American Dollar.}

\begin{itemize}
	\item $\mathbf{H_1}$ the set of \emph{Low Income} users, with an income less or equal than \$6300 a month. \\
	\item $\mathbf{H_2}$ the set of \emph{High Income} users, with income greater than \$6300 a month.
\end{itemize}

Getting the intersection of both datasets it's possible to create the \emph{Social Graph} $G$, with the same users of the set of \emph{Call Detail Records}, but where a subset of users also contain banking information from the set $B$, as shown in \cref{subsec:banktencomathing}.
This set, after being cleaned of outliers in \cref{subsec:outlier_filtering}, is the main input of the methods used in the thesis.

\section{Similar Studies}

This thesis is not the first socioeconomic study done in this dataset.

\Cref{sec:leo_correlations} presents a summary of the work done by Léo, Karsai, et\ al\. to correlate different consumption patterns with the socioeconomic level of these users.
In it, the authors present a model of \emph{Homophily} of calls between users of the same socioeconomic level, and proves that many different properties of the \emph{Social Graph} follow these correlations.

\Cref{Luo2017inferring} does a study similar to the previous one, where the author finds a correlation between users in the same socioeconomic level in social network patterns, mainly in the diversity of users in their links.
The study also presents several ways of preventing bias in these comparisons.

