\section{Accumulated Graph Features}
\label{sec:accumulatedfeatures}

\subsection{User Data}
\label{subsec:user_data}

The first accumulated features consist of accumulating the three quantifiable features described in Section~\ref{sec:graphfeatures} for every node, separated on whether those features are incoming or outgoing.

Additionally, we add two features corresponding to the \emph{In-Degree} and \emph{Out-Degree} of each node. This can be seen as the sum of an imaginary feature on each link, \textbf{Contacts}, which is always exactly $1$ when the link exists.

These features are defined mathematically for each node $v \in V$ in Equation~\ref{eq:user_data}.

\begin{equation}
\begin{gathered}
\begin{aligned}
\incalls_v &= \sum_{\substack{e \in E \\ e_d = v}} \calls_e &
\outcalls_v &= \sum_{\substack{e \in E \\ e_o = v}} \calls_e \\
\intime_v &= \sum_{\substack{e \in E \\ e_d = v}} \etime_e &
\outtime_v &= \sum_{\substack{e \in E \\ e_o = v}} \etime_e \\
\insms_v &= \sum_{\substack{e \in E \\ e_d = v}} \sms_e &
\outsms_v &= \sum_{\substack{e \in E \\ e_o = v}} \sms_e \\
\end{aligned} \\
\begin{aligned}
\incontacts_v &= \left| \left\{ e \in E \mid e_d = v \right\} \right| \\
\outcontacts_v &= \left| \left\{ e \in E \mid e_o = v \right\} \right|
\end{aligned}
\end{gathered}
\label{eq:user_data}
\end{equation}

These accumulated patterns present interesting distributions which are similar in all telcos \todo{Citation Needed}. These distributions are presented in Figure~\ref{fig:callsms}, Figure~\ref{fig:time}, and Figure~\ref{fig:contacts}.

\begin{figure}
\includegraphicsmaybe{figures/callsms_dist.png}
\caption{Distribution of the amount of outgoing calls and SMS for each user.}
\label{fig:callsms}
\end{figure}

\begin{figure}
\includegraphicsmaybe{figures/time_dist.png}
\caption{Distribution of total call time of outgoing calls for each user.}
\label{fig:time}
\end{figure}

\begin{figure}
\includegraphicsmaybe{figures/contact_dist.png}
\caption{Distribution of \emph{Out-Degree} for each user.}
\label{fig:contacts}
\end{figure}

\subsection{Higher Order User Data}

The features described in Section~\ref{subsec:user_data} correspond to the information about calls and SMS from a user $v \in V$ towards all of its neighbours. However, there's no reason why this information can't be extended to nodes at a higher distance from $v$.

The \emph{Ego Network} of the node $v$ is defined as the graph consisting of $v$ and its neighbors. A simple way to get more features about that node is to accumulate the call and SMS information about the edges which are \textbf{not} part of the \emph{Ego Network}, but one endpoint on the border\maybe{Formal definition of border of ego network?} of this.

Additionally, if we define the distance between two nodes using the intuitive definition which is presented on Equation~\ref{eq:distance}, we can define the \emph{User Data of Order $n$}, for any natural number $n$, as the accumulation of call and SMS information where one endpoint is on the border of the \emph{Ego Network of Order $n$}, and the other one isn't. The \emph{Ego Network of Order $n$} of a certain node $v$ is the subgraph composed of the node $v$, plus all the nodes which are at most at distance $n$ of $v$.

\begin{equation}
d \left( a, b \right) =
\begin{cases}
	0 & \text{if } a = b \\
	1 + \min_{v \in \ego \left(b \right)} d \left( a, v \right) & \text{otherwise}
\end{cases}
\label{eq:distance}
\end{equation}

This definition can be seen intuitively in Figure~\ref{fig:higherorderuserdata}. For reference purposes, we assign \emph{Level 0} to the information to the regular \emph{User Data} from Section~\ref{subsec:user_data}, while the user data from the \emph{Ego Network of Order $n$} is assigned \emph{Level $n$}\footnotemark{}.

\footnotetext{Note that the first definition isn't necessary; we could just say that the \emph{User Data of Order 0} contains information about the edges adjacent to \emph{Ego Network of Order 0}, which are the edges user in the regular User Data.}

\begin{figure}
\centering
\includegraphicsmaybe{figures/higherorderuserdata.png}
\caption{Example of the edges present in the calculation of the \emph{Higher Order User Data} for a certain node.}
\label{fig:higherorderuserdata}
\end{figure}
