\section{Accumulated Graph Features}
\label{sec:accumulatedfeatures}

\subsection{User Data}

The first accumulated features consist of accumulating the three quantifiable features described in Section~\ref{sec:graphfeatures} for every node, separated on whether those features are incoming or outgoing.

Additionally, we add two features corresponding to the \emph{In-Degree} and \emph{Out-Degree} of each node. This can be seen as the sum of an imaginary feature on each link, \textbf{Contacts}, which is always exactly $1$ when the link exists.

These features are defined mathematically for each node $v \in V$ in Equation~\ref{eq:user_data}.

\begin{equation}
\begin{gathered}
\begin{aligned}
\incalls_v &= \sum_{\substack{e \in E \\ e_d = v}} \calls_e &
\outcalls_v &= \sum_{\substack{e \in E \\ e_o = v}} \calls_e \\
\intime_v &= \sum_{\substack{e \in E \\ e_d = v}} \etime_e &
\outtime_v &= \sum_{\substack{e \in E \\ e_o = v}} \etime_e \\
\insms_v &= \sum_{\substack{e \in E \\ e_d = v}} \sms_e &
\outsms_v &= \sum_{\substack{e \in E \\ e_o = v}} \sms_e \\
\end{aligned} \\
\begin{aligned}
\incontacts_v &= \left| \left\{ e \in E \mid e_d = v \right\} \right| \\
\outcontacts_v &= \left| \left\{ e \in E \mid e_o = v \right\} \right|
\end{aligned}
\end{gathered}
\label{eq:user_data}
\end{equation}

These accumulated patterns present interesting distributions which are similar in all telcos \todo{Citation Needed}. These distributions are presented in Figure~\ref{fig:callsms}, Figure~\ref{fig:time}, and Figure~\ref{fig:contacts}.

\begin{figure}
\includegraphicsmaybe{figures/callsms_dist.png}
\caption{Distribution of the amount of outgoing calls and SMS for each user.}
\label{fig:callsms}
\end{figure}

\begin{figure}
\includegraphicsmaybe{figures/time_dist.png}
\caption{Distribution of total call time of outgoing calls for each user.}
\label{fig:callsms}
\end{figure}

\begin{figure}
\includegraphicsmaybe{figures/contact_dist.png}
\caption{Distribution of \emph{Out-Degree} for each user.}
\label{fig:callsms}
\end{figure}
