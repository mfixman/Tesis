\section{Introduction}

Mobile phone datasets present a rich view into social interactions and physical movements of large segments of the population. Thanks to both the exponential growth of mobile telecommunications and in the ability to store and process information, we can infer economic and social information from the telecommunications graph with a lot higher accuracy than what was possible just a few years ago.

The voice calls and text messages exchanged between people, together with the locations of those messages allow us to construct a rich social graph which can give us interesting insights on the users' social fabric, detailing regular patterns and behaviour between those that have similar behaviour~\cite{gonzalez2008understanding, ponieman2013human, sarraute2015city}.

There is a significant amount of economic homophily in people's communication~\cite{fixmanasonam2016} which results largely from social stratification between populations of different purchasing power~\cite{leo2015socioeconomic}, along with other economic indicators, which is similar to the homophily and stratification seen on other user features in many similar graph, like between age levels~\cite{mcpherson2001birds} in the mobile phone communications graph~\cite{sarraute2014} as well as the Facebook graph~\cite{ugander2011anatomy}.

Additionally, and in part resulting from this stratification, there are different patterns of communication features between users of distinct socioeconomic level. Since parsing this kind of graphs is still a novelty, we present several methods of feature extraction using raw data and compare several metrics of supervised machine learning algorithms to predict \emph{Socioeconomic Level} given a small ground truth with just a few values.
