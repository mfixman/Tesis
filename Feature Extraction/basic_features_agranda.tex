\section{Basic Graph Features}
\label{sec:graphfeatures}

We represent the network as a directed graph $G = \left< V, E \right>$, where the nodes $V$ represent the users and the edges $E$ represent the communication links between them.

This graph is created from the data presented in \cref{sec:data_sources}: $V$ is simply the union of all the origin and destination numbers on the intersection of either $P$ or $S$ and the set of numbers from the telco, and $E$ contains one element for every pair of nodes in either direction, where the data is the accumulation of the number of calls, the total time of those calls, and the number of text messages.


A small subset of the nodes, $T \subseteq V$ contains the \emph{Ground Truth} of the data, which indicates whether that user is part of the group of users with \emph{High Income} or \emph{Low Income}. This data will be useful to train the predictors, test them, and also for generating some features as seen in \cref{subsec:categoricaluserdata}.

$E$ contains the accumulated data of the edges between nodes. Each element $e \in E$ contains information about the \emph{Origin} and \emph{Destination} users, in addition to the amount of \emph{Calls}, the total call \emph{Time}, and the amount of \emph{SMS}.
