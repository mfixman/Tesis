\section{Accumulated Graph Features}
\label{sec:accumulatedfeatures}

This section presents several ways of transforming data from the graph $G = \left< V, E \right>$ into individual features for each user $v \in V$.

The aggregations are classified into named levels depending on the transformation done to $G$, and they are merged with levels containing less information as specified in \cref{fig:mlrelationships}. The total amount of columns in each featureset is presented in \cref{tab:features}.

\begin{table}
\centering
\begin{tabular}{>{\bfseries}l r}
\toprule
Level & Features \\
\midrule
$\ego_1$ & \num{8}  \\
$\ego_2$ & \num{16} \\
$\ego_3$ & \num{24} \\
$\cat_1$ & \num{24} \\
$\cat_2$ & \num{48} \\
$\cat_3$ & \num{72} \\
\bottomrule
\end{tabular}
\caption{Amount of total features per level.}
\label{tab:features}
\end{table}

\begin{figure}
\centering
\resizebox{!}{.2\textheight}{%
	\framebox{%
		
\begin{tikzpicture}[>=latex,line join=bevel,]
%%
\begin{scope}
  \pgfsetstrokecolor{black}
  \definecolor{strokecol}{rgb}{1.0,1.0,1.0};
  \pgfsetstrokecolor{strokecol}
  \definecolor{fillcol}{rgb}{1.0,1.0,1.0};
  \pgfsetfillcolor{fillcol}
  \filldraw (0.0bp,0.0bp) -- (0.0bp,168.0bp) -- (90.0bp,168.0bp) -- (90.0bp,0.0bp) -- cycle;
\end{scope}
\begin{scope}
  \pgfsetstrokecolor{black}
  \definecolor{strokecol}{rgb}{1.0,1.0,1.0};
  \pgfsetstrokecolor{strokecol}
  \definecolor{fillcol}{rgb}{1.0,1.0,1.0};
  \pgfsetfillcolor{fillcol}
  \filldraw (0.0bp,0.0bp) -- (0.0bp,168.0bp) -- (90.0bp,168.0bp) -- (90.0bp,0.0bp) -- cycle;
\end{scope}
\begin{scope}
  \pgfsetstrokecolor{black}
  \definecolor{strokecol}{rgb}{1.0,1.0,1.0};
  \pgfsetstrokecolor{strokecol}
  \definecolor{fillcol}{rgb}{1.0,1.0,1.0};
  \pgfsetfillcolor{fillcol}
  \filldraw (0.0bp,0.0bp) -- (0.0bp,168.0bp) -- (90.0bp,168.0bp) -- (90.0bp,0.0bp) -- cycle;
\end{scope}
\begin{scope}
  \pgfsetstrokecolor{black}
  \definecolor{strokecol}{rgb}{1.0,1.0,1.0};
  \pgfsetstrokecolor{strokecol}
  \definecolor{fillcol}{rgb}{1.0,1.0,1.0};
  \pgfsetfillcolor{fillcol}
  \filldraw (0.0bp,0.0bp) -- (0.0bp,168.0bp) -- (90.0bp,168.0bp) -- (90.0bp,0.0bp) -- cycle;
\end{scope}
  \node (d0p) at (72.0bp,106.0bp) [draw,circle] {$\cat_1$};
  \node (d1p) at (72.0bp,62.0bp) [draw,circle] {$\cat_2$};
  \node (d2p) at (72.0bp,18.0bp) [draw,circle] {$\cat_3$};
  \coordinate (inv2) at (18.0bp,18.0bp);
  \coordinate (inv0) at (72.0bp,150.0bp);
  \node (0) at (18.0bp,150.0bp) [draw,circle] {$\ego_1$};
  \node (1) at (18.0bp,106.0bp) [draw,circle] {$\ego_2$};
  \node (2) at (18.0bp,62.0bp) [draw,circle] {$\ego_3$};
  \definecolor{strokecolor}{rgb}{0.0,0.25,0.0};
  \draw [strokecolor,-stealth'] (2) ..controls (37.602bp,46.028bp) and (43.91bp,40.888bp)  .. (d2p);
  \definecolor{strokecolor}{rgb}{0.0,0.0,0.25};
  \draw [strokecolor,-stealth'] (d1p) ..controls (72.0bp,43.69bp) and (72.0bp,43.53bp)  .. (d2p);
  \definecolor{strokecolor}{rgb}{0.0,0.25,0.0};
  \draw [strokecolor,-stealth'] (1) ..controls (37.602bp,90.028bp) and (43.91bp,84.888bp)  .. (d1p);
  \definecolor{strokecolor}{rgb}{0.0,0.25,0.0};
  \draw [strokecolor,-stealth'] (0) ..controls (37.602bp,134.03bp) and (43.91bp,128.89bp)  .. (d0p);
  \definecolor{strokecolor}{rgb}{0.0,0.0,0.25};
  \draw [strokecolor,-stealth'] (d0p) ..controls (72.0bp,87.69bp) and (72.0bp,87.53bp)  .. (d1p);
  \definecolor{strokecolor}{rgb}{0.0,0.0,0.25};
  \draw [strokecolor,-stealth'] (0) ..controls (18.0bp,131.69bp) and (18.0bp,131.53bp)  .. (1);
  \definecolor{strokecolor}{rgb}{0.0,0.0,0.25};
  \draw [strokecolor,-stealth'] (1) ..controls (18.0bp,87.69bp) and (18.0bp,87.53bp)  .. (2);
%
\end{tikzpicture}


	}
}
\caption{Relationships between the \emph{Feature Extraction} methods of \cref{sec:accumulatedfeatures}. \textcolor{Blue}{Blue} edges represent a raise in \emph{Ego Network} size, a process which is describe in \cref{subsec:higherorderuserdata}, while \textcolor{ForestGreen}{green} edges represent adding label information, which is described in \cref{subsec:categoricaluserdata}}
\label{fig:mlrelationships}
\end{figure}

\subsection{User Data --- Level $\ego_1$}
\label{subsec:user_data}

The first accumulated features consist of accumulating the three quantifiable features described in \cref{sec:graphfeatures} for every node, separated on whether those features are incoming or outgoing.

\subsection{Higher Order User Data --- Level $\ego_{n > 1}$}

\label{subsec:higherorderuserdata}

The features described in \cref{subsec:user_data} correspond to the information about calls and SMS from a user $v \in V$ towards all of its neighbours. However, there's no reason why this information can't be extended to nodes at a higher distance from $v$.

The \emph{Ego Network} of the node $v$ is defined as the graph consisting of $v$ and its neighbors. A simple way to get more features about that node is to accumulate the call and SMS information about the edges which are \textbf{not} part of the \emph{Ego Network}, but one endpoint on the border of this.

It's possible to define the \emph{User Data of Order $n$}, for any natural number $n$, as the accumulation of call and SMS information for the nodes which are part of the \emph{Ego Network of Order $n$} and aren't part of the \emph{Ego Network of Order $n - 1$}. The \emph{Ego Network of Order $n$} of a certain node $v$ is the subgraph composed of the node $v$, plus all the nodes which are at most at distance $n$ of $v$.

For reference purposes, we assign the level $\ego_1$ to the information of the regular \emph{User Data} from \cref{subsec:user_data}, while the user data from the \emph{Ego Network of Order $n$} is assigned $\ego_n$ for some $n > 1$\footnotemark{}.

\footnotetext{Note that the first definition isn't necessary; we could just say that the \emph{User Data of Order 1} contains information about the edges adjacent to \emph{Ego Network of Order 1}, which are the edges user in the regular User Data. This is the reason why the \emph{User Data} has level $\ego_1$.}

\subsection{Categorical User Data --- Level $\cat_n$}
\label{subsec:categoricaluserdata}

Another approach to building features for the test is to combine the information contained in $E$, the list of edges, with the one in the ground truth $T \subseteq V$, which says whether a certain node represents a person of \emph{High Income} or a person of \emph{Low Income}.

\begin{table*}[t]
\begin{tabular*}{\textwidth}{>{\bfseries}l >{\bfseries}l >{\bfseries}{l @{\extracolsep{\fill}}} r r r r r r r r}
\toprule
\ct{Dataset} & \ct{Model} & \ct{Level} & \ct{Acc.} & \ct{Prec.} & \ct{Rec.} & \ct{AUC} & \ct{F\textsubscript{1}} & \ct{F\textsubscript{4}} & \ct{t\textsubscript{fit}} & \ct{t\textsubscript{pred}} \\
\midrule

\multirow{15}{*}{\centering Inner Graph}

& \multicolumn{2}{>{\bfseries}l}{Random}
& 0.499 & 0.499 & 0.500 & 0.499 & 0.500 & 0.500 & \ct{\NA} & \SI{0.005}{\second} \\

& \multicolumn{2}{>{\bfseries}l}{Majority}
& 0.681 & 0.640 & \textbf{0.826} & 0.681 & 0.721 & 0.712 & \ct{\NA} & \SI{0.059}{\second} \\

& \multicolumn{2}{>{\bfseries}l}{Bayesian\protect\footnotemark{}}

& 0.693 & 0.665 & 0.792 & \textbf{0.746} & \textbf{0.723} & \textbf{0.783} & \ct{\NA} & \SI{33.155}{\second} \\
\cmidrule{2-11}

& \multirow{5}{*}{LR} &
   $\ego_0$ & 0.536 & 0.531 & 0.625 & 0.536 & 0.574 & 0.619 & \SI{0.145}{\second}   & \SI{0.002}{\second} \\
&& $\ego_1$ & 0.535 & 0.525 & 0.730 & 0.535 & 0.611 & 0.714 & \SI{0.141}{\second}   & \SI{0.011}{\second} \\
&& $\ego_2$ & 0.568 & 0.578 & 0.525 & 0.569 & 0.550 & 0.528 & \SI{0.119}{\second}   & \SI{0.003}{\second} \\
&& $\cat_0$ & 0.686 & 0.655 & 0.785 & 0.686 & 0.714 & 0.776 & \SI{0.167}{\second}   & \SI{0.005}{\second} \\
&& $\cat_1$ & 0.693 & 0.665 & 0.780 & 0.693 & 0.718 & 0.772 & \SI{1.588}{\second}   & \SI{0.011}{\second} \\
&& $\cat_2$ & 0.693 & 0.670 & 0.764 & 0.692 & 0.714 & 0.758 & \SI{0.956}{\second}   & \SI{0.009}{\second} \\
\cmidrule{2-11}

& \multirow{5}{*}{RF} &
   $\ego_0$ & 0.548 & 0.548 & 0.550 & 0.548 & 0.549 & 0.550 & \SI{5.986}{\second}   & \SI{0.588}{\second} \\
&& $\ego_1$ & 0.582 & 0.583 & 0.577 & 0.582 & 0.580 & 0.577 & \SI{56.548}{\second}  & \SI{0.483}{\second} \\
&& $\ego_2$ & 0.576 & 0.577 & 0.580 & 0.576 & 0.579 & 0.580 & \SI{50.197}{\second}  & \SI{0.253}{\second} \\
&& $\cat_0$ & 0.671 & 0.665 & 0.690 & 0.671 & 0.677 & 0.688 & \SI{6.346}{\second}   & \SI{0.539}{\second} \\
&& $\cat_1$ & \textbf{0.714} & \textbf{0.713} & 0.716 & 0.714 & 0.714 & 0.716 & \SI{96.005}{\second}  & \SI{0.460}{\second} \\
&& $\cat_2$ & 0.709 & 0.710 & 0.711 & 0.709 & 0.711 & 0.711 & \SI{81.528}{\second}  & \SI{0.242}{\second} \\
\midrule

\multirow{14}{*}{\centering Outer Graph}

& \multicolumn{2}{>{\bfseries}l}{Random}
&       0.499 & 0.499 & 0.500 & 0.499 & 0.500 & 0.500 & \ct{\NA} & \SI{0.005}{\second} \\

& \multicolumn{2}{>{\bfseries}l}{Majority}
&       0.565 & 0.747 & 0.197 & 0.565 & 0.312 & 0.206 & \ct{\NA} & \SI{0.204}{\second} \\
\cmidrule{2-11}

& \multirow{5}{*}{LR} &
   $\ego_0$ & 0.534 & 0.586 & 0.234 & 0.534 & 0.335 & 0.243 & \SI{0.937}{\second}   & \SI{0.016}{\second} \\
&& $\ego_1$ & 0.547 & 0.617 & 0.250 & 0.547 & 0.356 & 0.260 & \SI{1.347}{\second}   & \SI{0.035}{\second} \\
&& $\ego_2$ & 0.563 & 0.586 & 0.430 & 0.563 & 0.496 & 0.437 & \SI{1.055}{\second}   & \SI{0.023}{\second} \\
&& $\cat_0$ & 0.565 & 0.746 & 0.198 & 0.565 & 0.313 & 0.207 & \SI{1.871}{\second}   & \SI{0.041}{\second} \\
&& $\cat_1$ & 0.577 & 0.727 & 0.247 & 0.577 & 0.368 & 0.257 & \SI{9.816}{\second}   & \SI{0.077}{\second} \\
&& $\cat_2$ & 0.589 & 0.636 & 0.415 & 0.589 & 0.503 & 0.424 & \SI{9.456}{\second}   & \SI{0.065}{\second} \\
\cmidrule{2-11}

& \multirow{5}{*}{RF} &
   $\ego_0$ & 0.543 & 0.544 & 0.529 & 0.543 & 0.536 & 0.530 & \SI{25.789}{\second}  & \SI{4.878}{\second} \\
&& $\ego_1$ & 0.578 & 0.585 & 0.537 & 0.578 & 0.560 & 0.540 & \SI{102.961}{\second} & \SI{5.608}{\second} \\
&& $\ego_2$ & 0.583 & 0.590 & 0.541 & 0.583 & 0.564 & 0.543 & \SI{70.447}{\second}  & \SI{3.148}{\second} \\
&& $\cat_0$ & 0.568 & 0.573 & 0.536 & 0.568 & 0.554 & 0.538 & \SI{32.981}{\second}  & \SI{5.371}{\second} \\
&& $\cat_1$ & 0.613 & 0.634 & 0.533 & 0.613 & 0.579 & 0.538 & \SI{44.911}{\second}  & \SI{6.002}{\second} \\
&& $\cat_2$ & 0.614 & 0.635 & 0.534 & 0.614 & 0.580 & 0.539 & \SI{50.589}{\second}  & \SI{3.484}{\second} \\
\bottomrule
\end{tabular*}
\caption{Resulting metrics of different methods used in Section~\ref{sec:results} and Section~\ref{sec:comparison} tested on both the \emph{Inner Graph} $\Upsilon$, which contains only nodes which have at least one neighbour with socioeconomic information, and the \emph{Outer Graph} $B^{\test}$, which includes all nodes. \textbf{Bolded} items represent the highest value for each metric.}
\label{tab:comparison}
\end{table*}


A simple way to do that is to do an approach similar to the \emph{User Data} presented in \cref{subsec:user_data}, but further discriminating each feature which corresponds to a node $v \in V$ and an edge $e \in E$ when $t \in T$ is the other endpoint of $e$ on whether $t$ corresponds to a person with high or low income. The resulting new features are of the form represented by the set in \cref{eq:matcatuserdata}.

\begin{equation}
\begin{Bmatrix} in \\ out \end{Bmatrix}
\times
\begin{Bmatrix} calls \\ time \\ sms \\ contacts \end{Bmatrix}
\times
\begin{Bmatrix} low \\ high \end{Bmatrix}
\label{eq:matcatuserdata}
\end{equation}

Creating these features naïvely will occur in overfitting, since the features are generated by data that's also used for training the supervised learning models. To solve this, the set $T$ is partitioned into two disjoint sets, $G$ and $H$, where $G$ contains roughly 0.75 of the nodes in $T$ is used to calculate the features, while $H$ contains the other 0.25 and is used to train the models.

It's possible to use the method presented in \cref{subsec:higherorderuserdata} generate an \emph{Ego Network of level $n$} of a certain node $v$, and accumulate the adjacent nodes to that network by socioeconomic level before accumulating their data. We refer to the level of these features as $\cat_n$.
