\documentclass[usenames,dvipsnames]{beamer}

\mode<presentation> {%
	\usetheme{Frankfurt}
}

% \usepackage{times}
\usepackage{amsmath,amssymb}
\usepackage[english]{babel}
\usepackage[utf8]{inputenc}
% \usepackage[latin1]{inputenc}

\usepackage{fancybox}
\usefonttheme[onlymath]{serif}
\boldmath

\usepackage{scalefnt}
\usepackage{ragged2e}
\usepackage{graphics}
\usepackage{url}
\usepackage{graphicx}
\usepackage{mathrsfs}
\usepackage{amsfonts}
\usepackage[authoryear,round]{natbib}
\usepackage{multirow}

\usepackage[group-separator={,}]{siunitx}
\usepackage{caption}
\usepackage{pbox}
\usepackage{tabularx}
\usepackage{booktabs}


% \usepackage{color}
% \usepackage[dvipsnames]{xcolor}
% \usepackage[x11names, rgb, dvipsnames]{xcolor}

\graphicspath{{../figures/}{../figures/workshop_data_science/}}

\usepackage{tikz}
\usetikzlibrary{shapes}
\usetikzlibrary{arrows}


\captionsetup[figure]{labelformat=empty}% redefines the caption setup of the figures environment in the beamer class.

\DeclareMathOperator{\ypred}{y_{pred}}
\DeclareMathOperator{\ytrue}{y_{true}}

\DeclareMathOperator{\rank}{rank}
\DeclareMathOperator{\cov}{cov}

\DeclareMathOperator{\TP}{TP}
\DeclareMathOperator{\TN}{TN}
\DeclareMathOperator{\FP}{FP}
\DeclareMathOperator{\FN}{FN}
\DeclareMathOperator{\FPR}{FPR}
\DeclareMathOperator{\AUC}{AUC}
\DeclareMathOperator{\TNR}{TNR}
\DeclareMathOperator{\TNA}{TNA}
\DeclareMathOperator{\TPA}{TPA}
\DeclareMathOperator{\TPR}{TPR}

\DeclareMathOperator{\Precision}{Precision}
\DeclareMathOperator{\Recall}{Recall}
\DeclareMathOperator{\InvPrecision}{Inverse\ Precision}
\DeclareMathOperator{\InvRecall}{Inverse\ Recall}
\DeclareMathOperator{\Accuracy}{Accuracy}

\DeclareMathOperator{\calls}{calls}
\DeclareMathOperator{\etime}{time}
\DeclareMathOperator{\sms}{sms}
\DeclareMathOperator{\contacts}{contacts}

\DeclareMathOperator{\ein}{in}
\DeclareMathOperator{\out}{out}

\DeclareMathOperator{\low}{low}
\DeclareMathOperator{\high}{high}

\DeclareMathOperator{\train}{train}
\DeclareMathOperator{\test}{test}

\DeclareMathOperator{\Betainc}{\Beta_{\operatorname{inc}}}

\DeclareMathOperator{\incalls}{incalls}
\DeclareMathOperator{\outcalls}{outcalls}
\DeclareMathOperator{\insms}{insms}
\DeclareMathOperator{\outsms}{insms}
\DeclareMathOperator{\intime}{outtime}
\DeclareMathOperator{\outtime}{outtime}
\DeclareMathOperator{\incontacts}{incontacts}
\DeclareMathOperator{\outcontacts}{outcontacts}

\DeclareMathOperator{\incallslow}{incallslow}
\DeclareMathOperator{\outcallslow}{outcallslow}
\DeclareMathOperator{\insmslow}{insmslow}
\DeclareMathOperator{\outsmslow}{insmslow}
\DeclareMathOperator{\intimelow}{outtimelow}
\DeclareMathOperator{\outtimelow}{outtimelow}
\DeclareMathOperator{\incontactslow}{incontactslow}
\DeclareMathOperator{\outcontactslow}{outcontactslow}

\DeclareMathOperator{\incallshigh}{incallshigh}
\DeclareMathOperator{\outcallshigh}{outcallshigh}
\DeclareMathOperator{\insmshigh}{insmshigh}
\DeclareMathOperator{\outsmshigh}{insmshigh}
\DeclareMathOperator{\intimehigh}{outtimehigh}
\DeclareMathOperator{\outtimehigh}{outtimehigh}
\DeclareMathOperator{\incontactshigh}{incontactshigh}
\DeclareMathOperator{\outcontactshigh}{outcontactshigh}

\DeclareMathOperator{\neigh}{Neigh}

\DeclareMathOperator{\dir}{dir}
\DeclareMathOperator{\cat}{cat}

\DeclareMathOperator{\ego}{ego}

\DeclareMathOperator{\logit}{logit}


% \DeclareMathOperator{\ego}{Ring}
% \DeclareMathOperator{\cat}{Cat}

\newcommand{\Beta}{B}

\newcommand{\mathA}{\mathcal{A}}
\newcommand{\mathB}{\mathcal{B}}
\newcommand{\mathC}{\mathcal{C}}
\newcommand{\mathG}{\mathcal{G}}
\newcommand{\mathP}{\mathcal{P}}

\newcommand{\TP}{\operatorname{TP}}
\newcommand{\TN}{\operatorname{TN}}
\newcommand{\FP}{\operatorname{FP}}
\newcommand{\TPR}{\operatorname{TPR}}
\newcommand{\FPR}{\operatorname{FPR}}
\newcommand{\AUC}{\operatorname{AUC}}

\newcommand{\closeopen}[2]{\left[ #1, #2 \right)}

\usepackage{array}
\newcolumntype{L}[1]{>{\raggedright\let\newline\\\arraybackslash\hspace{0pt}}m{#1}}
\newcolumntype{C}[1]{>{\centering\let\newline\\\arraybackslash\hspace{0pt}}m{#1}}
\newcolumntype{R}[1]{>{\raggedleft\let\newline\\\arraybackslash\hspace{0pt}}m{#1}}

\beamertemplatenavigationsymbolsempty

%%%%%%%%%%%%%%%%%%%%%%%%%%%%%%%%%%%%%%%%%%%%%%%%%%%%%%%%%%%%%%%%%%%%%%%%%%%%%%%%%%%%%%%%%%%%%%%%%%%%%%%%%%%%
%%%%%%%%%%%%%%%%%%%%%%%%%%%%%%%%%%%%%%%%%%%%%%%%%%%%%%%%%%%%%%%%%%%%%%%%%%%%%%%%%%%%%%%%%%%%%%%%%%%%%%%%%%%%

\title[Income Inference]
{Featurization methods and predictors for Income Inference based on Communication Patterns}

\author[Fixman et al.]{%
	\textbf{Carlos~Sarraute}\inst{1} \and
	Martin~Fixman\inst{1,2}\and
	Martin~Minnoni\inst{1} \and
	Matias~Travizano\inst{1} 
}

\institute{%
	\inst{1}Grandata Labs, Buenos Aires \& San Francisco \\
	\inst{2}Universidad de Buenos Aires, Argentina \\
	\texttt{mfixman@gmail.com, \{martin, mat, charles\}@grandata.com}
}

\bigskip
\bigskip

\date[NetSci2018]{NetSci 2018 \\ June 13-15, 2018}


%\setbeamertemplate{footline}[text line]{\bf \insertshortauthor \hfill \insertshorttitle \hfill \insertframenumber/29}
\setbeamertemplate{footline}[text line]{\hfill \insertframenumber / 24}

\AtBeginSection[]
{%
  \begin{frame}<beamer>{Agenda}
    \tableofcontents[currentsection]
  \end{frame}
}
  
%%%%%%%%%%%%%%%%%%%%%%%%%%%%%%%%%%%%%%%%%%%%%%%%%%%%%%%%%%%%%%%%%%%%%%%%%%%%%%%%%%%%%%%%%%%%%%%%%%%%%%%%%%%%
%%%%%%%%%%%%%%%%%%%%%%%%%%%%%%%%%%%%%%%%%%%%%%%%%%%%%%%%%%%%%%%%%%%%%%%%%%%%%%%%%%%%%%%%%%%%%%%%%%%%%%%%%%%%
%%%%%%%%%%%%%%%%%%%%%%%%%%%%%%%%%%%%%%%%%%%%%%%%%%%%%%%%%%%%%%%%%%%%%%%%%%%%%%%%%%%%%%%%%%%%%%%%%%%%%%%%%%%%
\begin{document}


\begin{frame}
\titlepage
\end{frame}



%%%%%%%%%%%%%%%%%%%%%%%%%%%%%%%%%%%%%%%%%%%%%%%%%%%%%%%%%%%%%%%%%%%%%%%%%%%%%%%%%%%%%%%%%%%%%%%%%%%%%%%%%%%%
\section{Introduction}
%%%%%%%%%%%%%%%%%%%%%%%%%%%%%%%%%%%%%%%%%%%%%%%%%%%%%%%%%%%%%%%%%%%%%%%%%%%%%%%%%%%%%%%%%%%%%%%%%%%%%%%%%%%%

\begin{frame}{Presentation}

\begin{block}{About Grandata}
\begin{itemize}

\item Founded in 2012

\medskip
\item Research and Data Science team:
\begin{itemize}
\item 10 persons based in Buenos Aires \& San Francisco
\item Mostly PhDs from University of Buenos Aires \\ (Computer Sc., Physics, Mathematics)
\end{itemize}

\medskip
\item We investigate \textbf{Human Dynamics}
\begin{itemize}
\item Use ``Big Data'' to analyze social networks and human behavior.
\item Integrate banking and mobile phone data.
\item Generate insights and predict user actions.
\end{itemize}

\end{itemize}

\end{block}
\end{frame}



\begin{frame}{Our Scientific Collaborations}
\begin{block}{Scientific Partners}
\begin{itemize}
\item Marton Karsai (ENS Lyon)
\medskip
\item Maria Oskarsdottir, Bart Baesens (KU Leuven)
\medskip
\item Hernan Makse (City College of New York) 
\medskip
\item Sandy Pentland and Human Dynamics team (MIT)
\item Marta Gonzalez and Human Mobility team (MIT)
\medskip
\item Fundación Mundo Sano, Fundación Bunge \& Born
\end{itemize}
\end{block}


{\tiny 
\nocite{leo2015socioeconomic}
\nocite{sarraute2015city}
\nocite{sarraute2014}
}
\end{frame}


\begin{frame}{Summary}

\begin{itemize}

\begin{block}{Objective}
Compare methods for \textbf{inference of socioeconomic status} \\
in a communication graph.
\end{block}
\medskip

\pause

\item Use 2 anonymized data sources:
\begin{itemize}
\item Call Detail Records (CDRs) from the operator allow us to construct a social graph.
\item Banking reported income for subset of clients obtained from a large bank.
\end{itemize} 

\item We construct an \textcolor{blue}{inference algorithm} that allows us to predict the socioeconomic status of users.

\item We compare it with standard machine learning techniques using features from nodes and their network.

\end{itemize}

\end{frame}

\begin{frame}{Datasets}

\begin{block}{Mobile Phone Data Source}

Each CDR \( p \in \mathP \) contains:
\begin{itemize}
\item phone numbers of origin and destination \( \left< p_o, p_d \right> \) \\
 anonymized using a cryptographic hash function
\item starting time \( p_t \), call duration \( p_s \)
\item latitude and longitude of antenna used \( \left< p_y, p_x \right> \) for subset of data.
\end{itemize} 

\end{block}

\pause

\begin{block}{Banking Information}

\begin{itemize}
\item Account balances for over 10 million clients of a bank for a period of 6 months, denoted \( \mathB \). 
\item For client \( b \in \mathB \) we have phone number \( b_p \), anonymized with same hash function used in \( \mathP \).
\item The average income of 6 months \( b_s \).

\end{itemize}


\end{block}

\end{frame}


\begin{frame}{Bank and Telco Matching}

\begin{itemize}

\item Phone numbers $ p_o $ and $ p_d $ are anonymized with same hash function as phone number in bank data $ b_p $.

\medskip
\item We can match users to unique phone to create social graph:
$$ G = \mathP \bowtie_{_{p_o = b_p}} \mathB \bowtie_{_{p_d = b_p}} \mathB $$

\medskip
\item This graph represents a total of \\
% \begin{itemize}
\qquad \num{2027554} nodes \\
\qquad \num{5044976} edges \\
\qquad \num{29599762} calls \\
\qquad \num{5476783} text messages.
% \end{itemize}
\end{itemize}

\end{frame}



%%%%%%%%%%%%%%%%%%%%%%%%%%%%%%%%%%%%%%%%%%%%%%%%%%%%%%%%%%%%%%%%%%%%%%%%%%%%%%%%%%%%%%%%%%%%%%%%%%%%%%%%%%%%
\section{Income Prediction}
%%%%%%%%%%%%%%%%%%%%%%%%%%%%%%%%%%%%%%%%%%%%%%%%%%%%%%%%%%%%%%%%%%%%%%%%%%%%%%%%%%%%%%%%%%%%%%%%%%%%%%%%%%%%

\begin{frame}{What do we use?}

\centering

\begin{LARGE}
Individual features~?

\bigskip

or

\bigskip
\medskip

Network topology~??
\end{LARGE}
\end{frame}


\begin{frame}{Income Homophily}

\begin{figure}[h]
\begin{center}
\includegraphics[width=0.7\columnwidth]{Homophily_income_origin_target_usd.png}
\caption{Number of calls between users, according to their monthly income \\
\medskip
{\small Similar to homophily with respect to age in~\cite{brea2014}.}
}
\label{homophily_heatmap}
\end{center}
\end{figure}
\end{frame}

\begin{frame}{What do we predict?}

Instead of predicting the exact value of a user's income, our strategy is to distinguish between 2 categories:
\begin{itemize}
\item $R_1 = \closeopen{1000}{6300}$ i.e.\ low income
\item $R_2 = \closeopen{6300}{\infty}$ i.e.\ high income
\end{itemize} 

\bigskip

We place users into two distinct groups $ H_1, H_2 \subseteq G$ of equal size:
\[
	g \in H_i \iff g_s \in R_i
\]

\end{frame}


%%%%%%%%%%%%%%%%%%%%%%%%%%%%%%%%%%%%%%%%%%%%%%%%%%%%%%%%%%%%%%%%%%%%%%%%%%%%%%%%%%%%%%%%%%%%%%%%%%%%%%%%%%%%
\section{Bayesian Predictor}
%%%%%%%%%%%%%%%%%%%%%%%%%%%%%%%%%%%%%%%%%%%%%%%%%%%%%%%%%%%%%%%%%%%%%%%%%%%%%%%%%%%%%%%%%%%%%%%%%%%%%%%%%%%%



\begin{frame}{Motivation}

\begin{columns}
\begin{column}{0.49\textwidth}

\begin{figure}[h]
\begin{center}
\includegraphics[width=\columnwidth]{distribucion1.png}
\end{center}
\end{figure}
\end{column}

\begin{column}{0.49\textwidth}

\begin{figure}[h]
\begin{center}
\includegraphics[width=\columnwidth]{distribucion2.png}
\end{center}
\end{figure}

\end{column}
\end{columns}

\medskip

The frequency of calls (to category 1 and 2) loses information.

We want to compare distributions.

\end{frame}





\begin{frame}{Determining the category}

	\begin{columns}
		\begin{column}{0.49\textwidth}

\begin{figure}[h]
\begin{center}
\includegraphics[width=\columnwidth]{distribucion3.png}
% \caption{Tres.}
\end{center}
\end{figure}
\end{column}

\begin{column}{0.49\textwidth}

\begin{figure}[h]
\begin{center}
\includegraphics[width=\columnwidth]{distribucion4.png}
% \caption{Cuatro.}
\end{center}
\end{figure}

\end{column}
\end{columns}


\begin{itemize}

\item Find the lowest 5 percentile $q5$ for this probability. 

\item If $q5$ is above threshold $\tau$, we assign user to $H_2$.

\item Take into account both the mean and the broadness (uncertainty) of the distribution. 

\item Category assigned to a user depends on its Beta distribution and on our choice of $\tau$.
\end{itemize}

\end{frame}


\begin{frame}{Accuracy}

\begin{figure}[p]
\begin{center}
\includegraphics[width=0.7\columnwidth]{accuracy_vs_fpr.png}
\caption{Accuracy as a function of FPR}
\label{fig:accuracy_vs_fpr}
\end{center}
\end{figure}

The best accuracy obtained is \num{0.71} for $\tau = 0.4$.

\end{frame}


\begin{frame}{ROC Curve}

\begin{figure}
\begin{center}
\includegraphics[width=0.7\columnwidth]{ROC_Beta_based_approach_201504.png}
\caption{ROC curve for prediction procedure}
\label{ROC_multiclass}
\end{center}
\end{figure}

We observed an $\AUC = 0.74$ indicating that our predictor is better than a random predictor ($\AUC \simeq 0.50$).

\end{frame}





%%%%%%%%%%%%%%%%%%%%%%%%%%%%%%%%%%%%%%%%%%%%%%%%%%%%%%%%%%%%%%%%%%%%%%%%%%%%%%%%%%%%%%%%%%%%%%%%%%%%%%%%%%%%
\section{Featurization Methods}
%%%%%%%%%%%%%%%%%%%%%%%%%%%%%%%%%%%%%%%%%%%%%%%%%%%%%%%%%%%%%%%%%%%%%%%%%%%%%%%%%%%%%%%%%%%%%%%%%%%%%%%%%%%%


\begin{frame}{Generation of Graph Features}

For each link $e \in E$ in the graph we have:

\begin{itemize}
	\item \textbf{Origin} of the calls and SMS
	\item \textbf{Destination} of the calls and SMS
	\item \textbf{Calls}: total number of calls
	\item \textbf{Time}: total time (in seconds) of all the calls
	\item \textbf{SMS}: total amount of messages
\end{itemize}

\end{frame}


\begin{frame}{Higher order user data -- $\ego_{n \geq 1}$ }

\begin{figure}
\centering
\framebox[\columnwidth]{%
	
\begin{tikzpicture}[>=latex,line join=bevel,]
%%
\node (b4) at (58.264bp,80.697bp) [draw,ellipse] {};
  \node (b5) at (107.65bp,46.247bp) [draw,ellipse] {};
  \node (b6) at (134.09bp,46.247bp) [draw,ellipse] {};
  \node (b7) at (183.48bp,80.697bp) [draw,ellipse] {};
  \node (b0) at (183.48bp,124.78bp) [draw,ellipse] {};
  \node (b1) at (158.52bp,150.63bp) [draw,ellipse] {};
  \node (b2) at (107.65bp,159.23bp) [draw,ellipse] {};
  \node (b3) at (64.526bp,134.74bp) [draw,ellipse] {};
  \node (c11) at (64.398bp,30.901bp) [draw,ellipse] {};
  \node (c10) at (36.354bp,54.739bp) [draw,ellipse] {};
  \node (a1) at (92.698bp,86.739bp) [draw,ellipse] {};
  \node (a0) at (133.84bp,129.35bp) [draw,ellipse] {};
  \node (c9) at (21.176bp,85.884bp) [draw,ellipse] {};
  \node (a2) at (139.69bp,78.794bp) [draw,ellipse] {};
  \node (c3) at (140.7bp,18.0bp) [draw,ellipse] {};
  \node (c2) at (220.56bp,85.884bp) [draw,ellipse] {};
  \node (c1) at (205.39bp,54.739bp) [draw,ellipse] {};
  \node (c0) at (220.56bp,119.6bp) [draw,ellipse] {};
  \node (c7) at (205.39bp,150.74bp) [draw,ellipse] {};
  \node (c6) at (64.398bp,174.58bp) [draw,ellipse] {};
  \node (c5) at (101.04bp,187.48bp) [draw,ellipse] {};
  \node (c4) at (140.7bp,187.48bp) [draw,ellipse] {};
  \node (a3) at (152.17bp,91.719bp) [draw,ellipse] {};
  \node (c8) at (21.176bp,119.6bp) [draw,ellipse] {};
  \node (v) at (120.87bp,102.74bp) [draw,fill=gray,ellipse] {v};
  \draw [red,] (v) ..controls (135.03bp,84.728bp) and (135.11bp,84.621bp)  .. (a2);
  \draw [ForestGreen,] (b6) ..controls (138.26bp,28.429bp) and (138.28bp,28.341bp)  .. (c3);
  \draw [blue,] (a1) ..controls (80.328bp,107.82bp) and (76.815bp,113.8bp)  .. (b3);
  \draw [ForestGreen,] (b7) ..controls (166.4bp,55.662bp) and (157.67bp,42.867bp)  .. (c3);
  \draw [blue,] (a0) ..controls (121.21bp,143.76bp) and (120.45bp,144.63bp)  .. (b2);
  \draw [ForestGreen,] (b2) ..controls (87.533bp,166.37bp) and (84.387bp,167.49bp)  .. (c6);
  \draw [ForestGreen,] (b4) ..controls (41.841bp,97.922bp) and (37.585bp,102.39bp)  .. (c8);
  \draw [red,] (v) ..controls (142.36bp,95.175bp) and (142.44bp,95.145bp)  .. (a3);
  \draw [ForestGreen,] (b7) ..controls (199.9bp,97.922bp) and (204.16bp,102.39bp)  .. (c0);
  \draw [blue,] (a0) ..controls (147.71bp,141.31bp) and (147.8bp,141.39bp)  .. (b1);
  \draw [ForestGreen,] (b0) ..controls (195.28bp,138.77bp) and (195.36bp,138.87bp)  .. (c7);
  \draw [ForestGreen,] (b4) ..controls (61.028bp,58.262bp) and (61.616bp,53.483bp)  .. (c11);
  \draw [ForestGreen,] (b2) ..controls (103.48bp,177.05bp) and (103.46bp,177.14bp)  .. (c5);
  \draw [blue,] (a1) ..controls (74.544bp,83.554bp) and (74.414bp,83.531bp)  .. (b4);
  \draw [red,] (v) ..controls (130.88bp,123.29bp) and (130.94bp,123.4bp)  .. (a0);
  \draw [ForestGreen,] (b7) ..controls (195.28bp,66.71bp) and (195.36bp,66.613bp)  .. (c1);
  \draw [blue,] (a0) ..controls (156.36bp,127.28bp) and (160.96bp,126.85bp)  .. (b0);
  \draw [ForestGreen,] (b4) ..controls (46.458bp,66.71bp) and (46.377bp,66.613bp)  .. (c10);
  \draw [blue,] (a3) ..controls (169.4bp,85.652bp) and (169.52bp,85.612bp)  .. (b7);
  \draw [ForestGreen,] (b5) ..controls (87.533bp,39.109bp) and (84.387bp,37.993bp)  .. (c11);
  \draw [ForestGreen,] (b2) ..controls (123.17bp,172.5bp) and (124.91bp,173.98bp)  .. (c4);
  \draw [blue,] (a1) ..controls (99.753bp,67.634bp) and (100.57bp,65.415bp)  .. (b5);
  \draw [blue,] (a2) ..controls (136.61bp,60.878bp) and (136.59bp,60.759bp)  .. (b6);
  \draw [red,] (v) ..controls (100.85bp,91.371bp) and (100.71bp,91.289bp)  .. (a1);
  \draw [blue,] (a1) ..controls (110.72bp,69.108bp) and (116.32bp,63.636bp)  .. (b6);
  \draw [blue,] (a1) ..controls (98.712bp,115.9bp) and (101.69bp,130.32bp)  .. (b2);
  \draw [ForestGreen,] (b7) ..controls (201.88bp,83.271bp) and (202.17bp,83.312bp)  .. (c2);
  \draw [ForestGreen,] (b4) ..controls (39.861bp,83.271bp) and (39.566bp,83.312bp)  .. (c9);
  \draw [ForestGreen,] (b0) ..controls (201.88bp,122.21bp) and (202.17bp,122.17bp)  .. (c0);
%
\end{tikzpicture}


}
\caption{Edges present in calculation of \emph{concentric rings of user data} for node $v$. \\
\textcolor{red}{Red} edges represent \emph{Ring 1},\\
 \textcolor{blue}{Blue} edges represent \emph{Ring 2}, \\
 \textcolor{ForestGreen}{Green} edges represent \emph{Ring 3}.}
\label{fig:higherorderuserdata}
\end{figure}


\end{frame}


\begin{frame}{Categorized user data -- $\cat_{n \geq 1}$}

We add the information of income category of the other party to 
edges in $\ego_{n}$ obtaining $\cat_n$:

\begin{columns}
\begin{column}{0.6\textwidth}
\begin{equation*}
\begin{Bmatrix} in \\ out \end{Bmatrix}
\times
\begin{Bmatrix} calls \\ time \\ sms \\ contacts \end{Bmatrix}
\times
\begin{Bmatrix} low \\ high \\ unknown \end{Bmatrix}
\label{eq:matcatuserdata}
\end{equation*}
\end{column}
\begin{column}{0.4\textwidth}
\begin{table}[t]
% \centering
\begin{tabular}{>{\bfseries}l r}
\toprule
Level & Features \\
\midrule
$\ego_1$ & \num{8}  \\
$\ego_2$ & \num{16} \\
$\ego_3$ & \num{24} \\
$\cat_1$ & \num{24} \\
$\cat_2$ & \num{48} \\
$\cat_3$ & \num{72} \\
\bottomrule
\end{tabular}
\end{table}
\end{column}
\end{columns}

\end{frame}



\begin{frame}{Machine Learning methods}

We used \textbf{Logistic Regression (LR)} and \textbf{Random Forests (RF)}. 
\medskip

We performed a grid search to tune hyper-parameters and used 5-fold cross validation.  
\medskip

We measure:
\begin{itemize}
\item \textcolor{Plum}{AUC} = Area under the ROC curve.
\item \textcolor{Plum}{Accuracy} = $(\TP + \TN) / (\operatorname{P} + \operatorname{N})$
\item \textcolor{Plum}{Precision} = $\TP / (\TP + \FP)$
\item \textcolor{Plum}{Recall} = $\TP / \operatorname{P}$
\item \textcolor{Plum}{F1-score} = 
	$ 2 \cdot \frac{1}{ \frac{1}{\mbox{precision}} + \frac{1}{\mbox{recall}} } $
\end{itemize}

\end{frame}



\begin{frame}{Results}
\begin{table}
{\small 
\begin{tabular*}{\textwidth}{>{\bfseries}l >{\bfseries}l @{\extracolsep{\fill}}>{\hspace{2em}}r r r }
\toprule
Model & Level & AUC & F\textsubscript{1}-score \\
\midrule

\multicolumn{2}{>{\bfseries}l}{Random Selection}
& 0.499 & 0.500   \\

\multicolumn{2}{>{\bfseries}l}{Majority Voting}
& 0.681 & 0.721   \\

\multicolumn{2}{>{\bfseries}l}{\textcolor{blue}{Bayesian Algorithm}}
& \textcolor{blue}{\textbf{0.746}} & \textcolor{blue}{\textbf{0.723}}  \\
\midrule

\multirow{5}{*}{LR} &
\textcolor{gray}{$\ego_1$} & \textcolor{gray}{0.536} & \textcolor{gray}{0.574} \\
& \textcolor{gray}{$\ego_2$} & \textcolor{gray}{0.535} & \textcolor{gray}{0.611}  \\
& \textcolor{gray}{$\ego_3$} & \textcolor{gray}{0.569} & \textcolor{gray}{0.550} \\
& $\cat_1$ & 0.686 & 0.714 \\
& $\cat_2$ & \textbf{0.693} & \textbf{0.718} \\
& $\cat_3$ & 0.692 & 0.714 \\

\midrule

\multirow{5}{*}{RF} &
\textcolor{gray}{$\ego_1$} & \textcolor{gray}{0.548} & \textcolor{gray}{0.549} \\
& \textcolor{gray}{$\ego_2$} & \textcolor{gray}{0.582} & \textcolor{gray}{0.580} \\
& \textcolor{gray}{$\ego_3$} & \textcolor{gray}{0.576} & \textcolor{gray}{0.579} \\
& $\cat_1$ & 0.671 & 0.677 \\
& $\cat_2$ & \textbf{0.714} & \textbf{0.714} \\
& $\cat_3$ & 0.709 & 0.711 \\
\bottomrule
\end{tabular*}
\caption{Results on \emph{Inner Graph}, which contains only nodes which have at least one neighbour with socioeconomic information. \textbf{LR} corresponds to \emph{Logistic Regression} models, and \textbf{RF} to \emph{Random Forest} models.}
\label{tab:innergraphresults}
}
\end{table}


\end{frame}



\begin{frame}{References}
\justifying%
\scalefont{0.75}
\bibliographystyle{unsrtnat}
\bibliography{../Tesis/bibliography/sna}{}

\end{frame}

\begin{frame}{~}
\centering
\begin{huge}
Thank you!
\end{huge}

\bigskip
\bigskip
\begin{Large}

Carlos Sarraute

\medskip
\emph{charles@grandata.com}

\medskip
\emph{@ch4rleston}
\end{Large}

\end{frame}

\end{document}

